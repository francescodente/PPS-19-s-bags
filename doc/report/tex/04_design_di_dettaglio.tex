\section{Design di dettaglio}

%(scelte rilevanti, pattern di progettazione, organizzazione del codice -- corredato da pochi ma efficaci diagrammi)

%Il design di dettaglio "esplode" (dettaglia) l'architettura, ma viene concettualmente prima dell'implementazione, quindi non metteteci diagrammi ultra-dettagliati estratti dal codice, quelli vanno nella parte di implementazione eventualmente.

% Core
A fronte della \textbf{analisi del dominio} è stato individuato un nucleo di entità e funzionalità necessari ad ogni singolo gioco da realizzarsi tramite l'utilizzo della libreria.
%
Questo \texttt{core} rispecchia quasi completamente l'analisi presentata, l'unica differenza risiede nella definizione di \textit{pawn}, \textit{tile}, \textit{move} e \textit{board state} che sono generiche anziché essere definite come interfacce.
%
Il nucleo è la base attorno la quale sono progettati tutti gli altri moduli di questo software, presentati di seguito.

% Estensioni dello stato tramite type class
% Concetti del DSL (?)
% - usato per MoveGeneration/MoveExecution
% - MoveGeneration usano generators
% - MoveExecution usano actions
% - modifiers per eseguire costrutti iterativi/condizionali e che si basano sui chainables (type class) per astrarre il modo in cui vengono eseguite le iterazioni/condizioni
% - feature (per accedere allo stato nonostante il ruleset sia definito a livello intensionale e quindi non ha accesso a uno stato particolare)
\subsection{RuleSet e Dsl}

Si analizza ora il design di dettaglio relativo al \texttt{RuleSet} e al suo \textbf{DSL}.

Il \texttt{RuleSet} idendifica l'insieme delle regole che definiscono se, in un determinato stato, una \textit{move} è valida e in quale modo questa modifichi lo stato del gioco al momento della sua esecuzione.
%
Per separare la logica del \texttt{RuleSet} dalle componenti che devono occuparsi della sua creazione si è fatto ricorso ad un \textbf{mixin} \texttt{RuleSetBuilder} da aggiungere al \texttt{RuleSet} per abilitare l'utilizzo del DSL nella sua definizione.

Il \texttt{RuleSetBuilder} % questo è da fare

\paragraph{MovesGeneration} fornisce gli strumenti per raccogliere l'insieme di tutti i generatori di \textit{move} per poi raccoglierli ed utilizzarli in ogni fase del gioco per generare la sequenza di tutte le mosse disponibili in un dato momento.

\paragraph{MovesExecution} è la componente che permette di definire in quale modo il \textit{GameState} vada aggiornato; mette a disposizione vari costrutti per aggiungere comportamenti e routine da eseguire al verificarsi di una data \textit{move} e permette, inoltre, di specificare azioni da eseguire prima e dopo l'esecuzione di mosse specifiche.

Sia \texttt{MovesGeneration} che \texttt{MovesExecution} forniscono le primitive per poter lavorare agilmente con il sistema del \textit{RuleSet}.
%
Questi però sono solamente un involucro e da soli non sono in grado di fornire una semplificazione al sistema per chi utilizzerà poi la libreria.
%
Grazie all'aggiunta di un \texttt{DSL} complesso mostrato in figura (FIGURA).

\subsection{Interaction}

Le funzionalità abilitanti l'interazione con i giocatori sono state incapsulate totalmente nel modulo \texttt{interaction}, che comprende \texttt{view} e \texttt{controller}, con un interfacciamento verso il \texttt{model}.

Un gioco completo sviluppato con la libreria prevede due viste principali:
\begin{itemize}
    \item \textit{Menu}: il punto di ingresso dell'applicazione dove poter selezionare opzioni come giocare una nuova partita oppure uscire dal programma;
    \item \textit{Game}: una partita in corso.
\end{itemize}
%
Ciascuna di queste è composta da una specifica coppia di \texttt{subView} e \texttt{subController}.
%

% - View
\subsubsection{View}
% - - View principale -> Gestione subview
La \texttt{View} principale è in grado di creare e inizializzare le \texttt{SubView} esponendo un metodo per ritornare ognuna di esse: questo permette di centralizzare le informazioni relative al tipo della \texttt{View} e consente di rimpiazzarla in blocco. %brutto
% - - Menu view
\paragraph{Menu view} 
%
È il componente per gestire tutto ciò che è precedente o successivo al \textit{Game} (e.g. creazione \textit{Game}, chiusura dell'applicazione).
%
Il \texttt{MenuView} permette tramite un'apposita UI di scegliere come procedere tra le varie opzioni.
% - - Game view
\paragraph{Game view} 
%
È in grado di presentare il \textit{Game} nel caso di una \textit{move} valida, e in caso contrario gestisce la \texttt{Failure}.
%
Nello specifico, tramite la composizioni di \texttt{Renderer}, la \texttt{GameView} presenta la grafica dello stato attuale del \textit{Game}.
% - - - Renderers
\subparagraph{Renderer}
%
Compongono le \texttt{GameView} ed espongono un metodo per disegnare una specifica caratteristica del \textit{Game}.
%
Richiamando l'aggiornamento di tutti i \texttt{Renderer} la \texttt{GameView} è in grado di mostare tutte le caratteristiche dello stato attuale del \textit{Game}.
% - - - Event
\subparagraph{Event}
%
Rappresentano un'interazione dell'utente con la \texttt{GameView} e un'insieme di \texttt{Event} può comporre una \textit{move}.
%
Una \texttt{GameView}, in quanto \texttt{Observable}, è in grado di notificare il proprio \texttt{GameController} degli \texttt{Event} che vengono generati. 
% - Controller
\subsubsection{Controller}
I controller presentano una dipendenza dalle view in quanto devono chiamare i metodi della specifica \texttt{SubView} per fornire il feedback appropriato rispetto all'interazione intrapresa dal giocatore.
%
% - - Application controller
L'\texttt{ApplicationController} è responsabile di gestire la \texttt{View} principale, aggiungendovi i \textbf{listener} relativi e facendola partire.
% - - Menu controller
\paragraph{Menu controller}
%
Presenta le funzionalità per gestire l'avvio di una partita e la terminazione dell'applicazione.
%
Questo controller è una prima interfaccia con il \textbf{model}, che nel caso di avvio di una partita viene utilizzato nella forma della \texttt{GameDescription} per la generazione di un nuovo \texttt{Game}.
%
Successivamente viene preparata la GameView con i parametri relativi, viene aggiunto un nuovo \texttt{GameController} come listener e si passa alla nuova vista.
% - - Game controller
\paragraph{Game controller}
%
Gestisce gli \texttt{Event} che gli vengono notificati, presentando un comportamento diverso in base alla situazione:
\begin{itemize}
    \item Se l'evento assieme a quelli precedentemente salvati corrisponde ad una \texttt{Move}, allora questa viene eseguita sul \texttt{Game} e gli eventi memorizzati vengono azzerati;
    \item Se l'evento assieme a quelli precedentemente salvati non corrisponde ad una \texttt{Move}, allora viene memorizzato;
    \item Se l'evento è \texttt{Quit} allora termina la \texttt{GameView}.
\end{itemize}
