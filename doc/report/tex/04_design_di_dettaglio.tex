\section{Design di dettaglio}

%(scelte rilevanti, pattern di progettazione, organizzazione del codice -- corredato da pochi ma efficaci diagrammi)

%Il design di dettaglio "esplode" (dettaglia) l'architettura, ma viene concettualmente prima dell'implementazione, quindi non metteteci diagrammi ultra-dettagliati estratti dal codice, quelli vanno nella parte di implementazione eventualmente.

% Estensioni dello stato tramite type class
% Concetti del DSL (?)
% - usato per MoveGeneration/MoveExecution
% - MoveGeneration usano generators
% - MoveExecution usano actions
% - modifiers per eseguire costrutti iterativi/condizionali e che si basano sui chainables (type class) per astrarre il modo in cui vengono eseguite le iterazioni/condizioni
% - feature (per accedere allo stato nonostante il ruleset sia definito a livello intensionale e quindi non ha accesso a uno stato particolare)
% Interaction
% - Definizione dei concetti di Application, Menu e Game
% - Game setup come entry point
% - View
% - - View principale -> Gestione subview
% - - Menu view
% - - Game view
% - - - Renderers
% - - - Eventi
% - - Cli
% - Controller
% - - Application controller
% - - Menu controller
% - - Game controller
