\section{Design di dettaglio}

%(scelte rilevanti, pattern di progettazione, organizzazione del codice -- corredato da pochi ma efficaci diagrammi)

%Il design di dettaglio "esplode" (dettaglia) l'architettura, ma viene concettualmente prima dell'implementazione, quindi non metteteci diagrammi ultra-dettagliati estratti dal codice, quelli vanno nella parte di implementazione eventualmente.

% Estensioni dello stato tramite type class
% Concetti del DSL (?)
% - usato per MoveGeneration/MoveExecution
% - MoveGeneration usano generators
% - MoveExecution usano actions
% - modifiers per eseguire costrutti iterativi/condizionali e che si basano sui chainables (type class) per astrarre il modo in cui vengono eseguite le iterazioni/condizioni
% - feature (per accedere allo stato nonostante il ruleset sia definito a livello intensionale e quindi non ha accesso a uno stato particolare)
\subsection{RuleSet e Dsl}

Si va ora ad analizzare il design di dettaglio relativo al \textit{RuleSet} e al suo \textit{DSL}.

Il \textit{RuleSet} idendifica l'insieme delle regole che definiscono se, in un determinato stato, una \textit{move} è valida e in quale modo questa modifichi lo stato del gioco al momento della sua esecuzione.
%
Per separare la logica del \textit{rule set} dalle componenti che devono occuparsi della sua creazione si è fatto ricorso ad un \textbf{mixin} \texttt{RuleSetBuilder} da aggiungere al \texttt{RuleSet} per abilitare l'utilizzo del DSL nella sua definizione.
%
Il \texttt{RuleSetBuilder} 

% parlare in questo contesto di tutto quello che riguarda la generazione delle mosse TODO
\subsubsection{MovesGeneration}

\subsubsection{MovesExecution}

\textit{MovesExecution} è la componente che gestisce l'esecuzione delle mosse e ha il compito di definire in quale modo lo stato viene modificato a fronte dell'invocazione di una \textit{move} valida.

\subsection{Interaction}

Le funzionalità abilitanti l'interazione con i giocatori sono state incapsulate totalmente nel modulo \texttt{interaction}, che comprende \textbf{view} e \textbf{controller}, con un interfacciamento verso il \textbf{model}.

Un gioco completo sviluppato con la libreria prevede due viste principali:
\begin{itemize}
    \item \textit{Menu}: il punto di ingresso dell'applicazione dove poter selezionare opzioni come giocare una nuova partita oppure uscire dal programma;
    \item \textit{Game}: una partita in corso.
\end{itemize}
%
Ciascuna di queste è composta da una specifica coppia di \texttt{subView} e \texttt{subController}.
%

% - View
\subsubsection{View}
% - - View principale -> Gestione subview
% - - Menu view
% - - Game view
% - - - Renderers
% - - - Eventi
% - - Cli
% - Controller
\subsubsection{Controller}
I controller presentano una dipendenza dalle view in quanto devono chiamare i metodi della specifica \texttt{subView} per fornire il feedback appropriato rispetto all'interazione intrapresa dal giocatore.
%
% - - Application controller
L'\texttt{ApplicationController} è responsabile di gestire la \texttt{View} principale, aggiungendovi i \textbf{listener} relativi e facendola partire.
% - - Menu controller
\paragraph{Menu controller}
%
Presenta le funzionalità per gestire l'avvio di una partita e la terminazione dell'applicazione.
%
Questo controller è una prima interfaccia con il \textbf{model}, che nel caso di avvio di una partita viene utilizzato nella forma della \texttt{GameDescription} per la generazione di un nuovo \texttt{Game}.
%
Successivamente viene preparata la GameView con i parametri relativi, viene aggiunto un nuovo \texttt{GameController} come listener e si passa alla nuova vista.
% - - Game controller
\paragraph{Game controller}
%
Gestisce gli \texttt{Event} che gli vengono notificati, presentando un comportamento diverso in base alla situazione:
\begin{itemize}
    \item Se l'evento assieme a quelli precedentemente salvati corrisponde ad una \texttt{Move}, allora questa viene eseguita sul \texttt{Game} e gli eventi memorizzati vengono azzerati;
    \item Se l'evento assieme a quelli precedentemente salvati non corrisponde ad una \texttt{Move}, allora viene memorizzato;
    \item Se l'evento è \texttt{Quit} allora termina la \texttt{GameView}.
\end{itemize}
