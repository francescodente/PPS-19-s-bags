\section{Requisiti}

%questa sezione deve essere sufficiente per uno sviluppatore per giungere ad un sistema che fa quello che fa il nostro

%Attenzione in particolare ai requirement non funzionali: 1) non siano troppo vaghi altrimenti sono inverificabili, e quindi praticamente inutili; 2) se il sistema è distribuito, è inevitable dire cosa vi aspettate in termini di di robustezza a cambiamenti/guasti (quali?, come?), e scalabilità (in quale dimensione? fino a che punto?).

%------------------------------------------------------------------------------------

\subsection{Business}

La libreria consentirà di realizzare giochi da tavolo con le seguenti caratteristiche:

\begin{itemize}
    \item Giochi basati sul movimento di \textit{pawn} all'interno di \textit{board};
    \item Giochi a informazione perfetta; % TODO add foot note https://it.wikipedia.org/wiki/Gioco_a_informazione_completa
    % TODO
\end{itemize}

In paricolare deve essere possibile realizzare i seguenti giochi:

\begin{itemize}
    \item ``\textbf{Put-In-Put-Out}'': un gioco dove l'utente può piazzare e rimuovere \textit{pawn} dalla \textit{board}, senza ulteriori funzionalità.
\end{itemize}

%------------------------------------------------------------------------------------
\subsection{Utente}

Uno sviluppatore che utilizzerà la libreria dovrà prima dare una specifica del gioco utilizzando i costrutti messi a disposizione dalla stessa, e successivamente potrà usufruire del gioco generato da tali specifiche fornendovi dei comandi appositi.

In particolare l'utente può definire i seguenti aspetti:

\begin{itemize}
    \item statici:
    \begin{itemize}
        \item rappresentazione dei \textit{pawn} supportati dal gioco
        \begin{itemize}
            \item fornendo il tipo base
        \end{itemize}
        \item rappresentazione delle \textit{tile} che compongono la \textit{board}
        \begin{itemize}
            \item fornendo il tipo
        \end{itemize}
        \item le \textit{move} che si possono effettuare
        \begin{itemize}
            \item specificando quale \textit{action} viene applicata
        \end{itemize}
        \item definizione del \textit{game state} iniziale.
    \end{itemize}

    \item dinamici:
    \begin{itemize}
        \item inizializzazione della partita
        \item applicazione delle \textit{move}
    \end{itemize}
\end{itemize}

%Per uno sviluppatore che utilizzerà la libreria sarà possibile specificare:

% \begin{itemize}
%     \item le componenti del gioco e le loro caratteristiche;
%     \item le regole del gioco:
%     \begin{itemize}
%         \item insieme delle \textit{move} disponibili;
%         \item condizioni che determinano la validità di una \textit{move};
%         \item conseguenze dell'esecuzione di una \textit{move} sul \textit{game state};
%     \end{itemize}
%     \item lo svolgimento del gioco, tra cui:
%     \begin{itemize}
%         \item lo stato iniziale;
%         \item le condizioni di avanzamento;
%         \item le condizioni di terminazione;
%     \end{itemize}
%     \item la modalità con cui l'input dell'utente finale viene convertito in \textit{move}. % TODO da sistemare
% \end{itemize}

\subsection{Funzionali}

% TODO intro della sezione
La libreria implementerà le funzionalità base che, opportunamente composte, creano un gioco con un determinato insieme di regole.

Allo stato attuale la libreria supporta:

\begin{itemize}
    \item un solo tipo di \textit{pawn}
    \item un solo tipo di \textit{tile}
    \item solo \textit{board} quadrate
    \item le \textit{action} di base per operare sulla \textit{board}, quali:
    \begin{itemize}
        \item inserimento di un \textit{pawn} su una \textit{tile}
        \item rimozione di un \textit{pawn} da una \textit{tile}
    \end{itemize}
    \item definizione di una \textit{move} come applicazione di una singola \textit{action} scelta fra quelle disponibili
    \item definizione del \textit{game state} iniziale come \textit{board} vuota.
\end{itemize}

Data la descrizione complessiva del gioco, la libreria fornirà le astrazioni per effettuare le seguenti operazioni:

\begin{itemize}
    \item inizializzazione di una partita, che crea il \textit{game state} di partenza
    \item fornita una \textit{move} questa verrà applicata al gioco in base al suo \textit{game state} corrente.
\end{itemize}

\subsection{Non funzionali}

% TODO intro della sezione


\subsection{Di implementazione}

La libreria dovrà essere sviluppata in scala, e la sua correttezza sarà verificata tramite testing automatizzato utilizzando \textbf{scalatest}.