\section{Requisiti}

Questa sezione è redatta considerando come utenti i programmatori che utilizzeranno la libreria per sviluppare i giochi, mentre i requisiti dei giocatori che utilizzeranno il software costruito adoperando questa libreria non sono considerati rilevanti.

\subsection{Business}

La libreria consentirà di realizzare giochi da tavolo con le seguenti caratteristiche:
%
\begin{enumerate}
    \item Requisiti di business.
    \begin{enumerate}[label*=\arabic*.]
        \item Giochi basati sul movimento di pedine all'interno di un tabellone;
        \item Giochi a informazione perfetta\footnote{\url{https://it.wikipedia.org/wiki/Gioco_a_informazione_completa}};
        \item Giochi in cui è necessario gestire lo sviluppo della partita (turni, condizioni di vittoria, vincitore della partita, \dots).
    \end{enumerate}
\end{enumerate}

In paricolare deve essere possibile realizzare i seguenti giochi:
%
\begin{enumerate}[resume]
    \item Giochi sviluppati.
    \begin{enumerate}[label*=\arabic*.]
        \item ``\textbf{Put-In-Put-Out}'': un gioco dove l'utente può piazzare e rimuovere una pedina sull'unica casella del tabellone, senza ulteriori funzionalità;
        %
        \item ``\textbf{TicTacToe}'': il gioco del tris, in cui a turno due giocatori piazzano in una casella di una matrice 3x3 il proprio simbolo (solitamente una X e una O).
        %
        Vince il giocatore che riesce a disporre tre delle proprie pedine in linea retta orizzontale, verticale o diagonale.
        %
        Se la matrice viene riempita senza che nessun sia riuscito a creare un tris, il gioco finisce in parità;
        %
        \item ``\textbf{Connect Four}'': Forza4 è un gioco basato su tabellone, in particolare una matrice 6x7, all'interno del quale si possono posizionare delle pedine solo nella prima casella libera di ogni colonna.
        %
        In particolare due giocatori, in modo alternato, inseriscono uno delle proprie pedine all'interno di una colonna.
        %
        Il vincitore è colui che riesce a disporre quattro delle proprie pedine adiacenti in linea retta orizzontale, verticale o diagonale.
        %
        Nel caso di riempimento del tabellone senza che nessun giocatore sia riuscito a vincere, il gioco finisce in parità;
        %
        \item ``\textbf{Othello}'': un gioco basato su una scacchiera 8x8 chiamata othelliera.
        %
        Il gioco prevede la presenza di due giocatori ai quali è assegnato un colore (solitamente bianco e nero). 
        %
        Il gioco inizia con due pedine per giocatore poste nelle caselle centrali a formare una 'X'.
        %
        I turni dei giocatori sono alternati; in ogni turno un giocatore può posizionare una sua pedina in modo da imprigionare una o più pedine dell'avversario tra quella posizionata ed altre sue pedine.
        %
        Ogni pedina imprigionata diventa del giocatore di turno. 
        %
        In mancanza di mosse disponibili il giocatore deve passare il turno.
        %
        L'obiettivo del gioco è di avere più pedine dell'avversario sull'othelliera quando non ci sono più mosse disponibili o quando l'othelliera è piena.
    \end{enumerate}
\end{enumerate}

%------------------------------------------------------------------------------------
\subsection{Utente}

Uno sviluppatore che utilizzerà la libreria dovrà fornire una specifica del gioco utilizzando i costrutti messi a disposizione dalla stessa, e successivamente potrà usufruire del gioco generato da tali specifiche tramite degli appositi comandi.
%
In particolare l'utente può definire i seguenti aspetti:
%
\begin{enumerate}[resume]
    \item Requisiti utente.
    \begin{enumerate}[label*=\arabic*.]
        \item rappresentazione delle pedine all'interno del gioco;
        \item rappresentazione delle celle e del modo in cui compongono il tabellone;
        \item rappresentazione delle mosse;
        \item rappresentazione dello stato del gioco;
        \item regole del gioco:
        \begin{enumerate}[label*=\arabic*.]
            \item con API classica o tramite DSL interno;
            \item generazione delle mosse disponibili a partire da un dato stato di gioco;
            \item definizione delle azioni che vengono eseguite sullo stato del gioco a fronte dell'esecuzione di una mossa;
        \end{enumerate}
        \item descrizione del gioco, che prevede:
        \begin{enumerate}[label*=\arabic*.]
            \item definizione dello stato del gioco iniziale;
            \item definizione di eventuali estensioni del gioco:
            \begin{enumerate}[label*=\arabic*.]
                \item giocatori che partecipano a una partita;
                \item flusso dei turni;
                \item condizioni di terminazione;
            \end{enumerate}
        \end{enumerate}
        \item interazione tramite UI, definendo:
        \begin{enumerate}[label*=\arabic*.]
            \item il tipo di UI utilizzata tra le seguenti:
            \begin{enumerate}[label*=\arabic*.]
                \item \label{req:cli} interfaccia testuale (CLI), fornita dalla libreria;
                \item \label{req:custom_view} interfaccia personalizzata, definita dall'utente;
            \end{enumerate}
            \item \label{req:renderers_parameters} i parametri necessari alla UI per effettuare il display dello stato del gioco;
            \item \label{req:events} i parametri per la conversione degli eventi generati dalla UI in mosse da eseguire sul gioco;
        \end{enumerate}
    \end{enumerate}
\end{enumerate}

\subsection{Funzionali}

La libreria implementerà le funzionalità base che, opportunamente composte, costituiranno un gioco con un determinato insieme di regole.

La libreria dovrà:
%
\begin{enumerate}[resume]
    \item Requisiti funzionali.
    \begin{enumerate}[label*=\arabic*.]
        \item supportare diversi tipi di pedine, a patto che abbiamo uno specifico sopratipo comune;
        \item supportare diversi tipi di caselle, a patto che abbiamo uno specifico sopratipo comune;
        \item \label{req:board_state} supportare diversi tipi di tabelloni:
        \begin{enumerate}[label*=\arabic*.]
            \item fornendo astrazioni di base per i tabelloni comuni (e.g. rettangolari, quadrati, \dots);
            \item lasciando la possibilità allo sviluppatore di rappresentare strutture personalizzate;
        \end{enumerate}
        \item fornire le seguenti azioni sul tabellone:
        \begin{enumerate}[label*=\arabic*.]
            \item inserimento di una pedina su una casella;
            \item rimozione di una pedina da una casella;
            \item lettura dello stato di una casella, ovvero quale eventuale pedina vi è posizionata sopra;
            \item azioni composite, derivate da una sequenza arbitraria delle precedenti;
        \end{enumerate}
        \item gestire lo svolgersi di una partita:
        \begin{enumerate}[label*=\arabic*.]
            \item permettendo di verificare in ogni momento se una mossa è valida;
            \item permettendo di ottenere l'insieme di tutte le mosse valide in un dato istante; 
            \item impedendo l'esecuzione di mosse illegali;
            \item aggiornando lo stato corrente all'esecuzione di una mossa valida, seguendo quanto indicato dalle regole;
            \item \label{req:end_game_cond} rilevando la terminazione della partita;
            \item \label{req:undo} permettendo di annullare una mossa eseguita, mantenendo una cronologia della partita;
        \end{enumerate}
        \item \label{req:dsl} fornire un DSL per la scrittura delle regole, che offre una sintassi semplificata e in linguaggio simil-naturale per:
        \begin{enumerate}[label*=\arabic*.]
            \item \label{req:action} eseguire azioni che modificano lo stato del gioco:
            \begin{enumerate}[label*=\arabic*.]
                \item inserimento di pedine;
                \item rimozione di pedine;
                \item spostamento di pedine;
                \item rimpiazzo di pedine;
                \item avanzamento del turno;
                \item altre azioni personalizzate;
            \end{enumerate}
            \item \label{req:generator} definire le regole per generare le mosse disponibili in un dato momento;
            \item concatenare le istruzioni di base, per creare azioni composite o sequenze di generatori;
            \item iterare sulle proprietà dello stato del gioco;
            \item verificare condizioni sullo stato del gioco;
            \item \label{req:iterating} innestare i costrutti iterativi e condizionali.
        \end{enumerate}
    \end{enumerate}
\end{enumerate}

\subsection{Non funzionali}

Data la natura del software prodotto, la sua dipendenza dal codice degli utenti che ne faranno utilizzo, e la mancanza di elementi distribuiti, non sono individuati requisiti non funzionali.
%
Questi sono lasciati agli utilizzatori della libreria che dovranno calibrarli in base al tipo di applicazione e di utente.

\subsection{Di implementazione}

La libreria dovrà essere sviluppata in Scala, e verranno svolti dei test con \textbf{Scalatest} per minimizzare la presenza di errori e facilitare l'aggiornamento di eventuali funzionalità.
