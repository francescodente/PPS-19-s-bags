\section{Processo di sviluppo}

% (modalità di divisione in itinere dei task, meeting/interazioni pianificate, modalità di revisione in itinere dei task, scelta degli strumenti di test/build/continuous integration)

Il processo di sviluppo adottato dal team è \textbf{incrementale} ed \textbf{iterativo}, fortemente ispirato a \textbf{Scrum} ma adattato alle esigenze di un progetto dove gli sviluppatori devono ricoprire anche i ruoli di \textbf{Scrum Master} e \textbf{Product Owner}.
%
Il team effettua \textbf{Sprint} settimanali in cui decide gli obiettivi, suddivide i compiti e svolge incontri quotidiani.
%
Alla fine viene discusso il lavoro portato a termine in vista dello Sprint successivo.

\subsection{Meeting}
Gli incontri sono un tassello fondamentale per le metodologie \textbf{agili}, ed il nostro processo riprende quelle di Scrum, con tutto il team che partecipa ad ogni incontro, svolgendo sempre il ruolo di sviluppatori e dove necessario quelli di Scrum Master e Product Owner.

\subsubsection{Sprint planning}
Questo incontro è svolto a cadenza settimanale all'inizio di ogni Sprint, e vengono discussi i seguenti punti:

\begin{itemize}
    \item Identificazione dell'\textbf{incremento} che si vuole ottenere alla fine dello Sprint tramite utilizzo e rifinimento del \textbf{Product Backlog};
    \item Definizione dei \textbf{Task} come unità di lavoro pratica per soddisfare i \textbf{requisiti};
    \item Assegnazione dei Task ai membri del team.
\end{itemize}

Lo Sprint Planning ha una durata massima di 2 ore.

\subsubsection{Daily Scrum}
Durante il Daily Scrum ogni sviluppatore espone al team i seguenti punti:

\begin{itemize}
    \item Quale lavoro ha svolto la giornata precedente;
    \item Quale lavoro intende svolgere nella giornata corrente;
    \item Eventuali possibili impedimenti per il lavoro da svolgere, e come gli altri membri del team potrebbero aiutare ad affrontare il problema.
\end{itemize}

La durata di questo incontro è al massimo di 15 minuti.

\subsubsection{Sprint review}
Lo Sprint review analizza il prodotto sviluppato con i seguenti punti:

\begin{itemize}
    \item Ispezione dell'incremento ottenuto, ovvero aggiornamento del team sullo stato di sviluppo complessivo del progetto;
    \item Adattamento Product Backlog;
    \item Valutazione di eventuali ritardi sulla tabella di marcia;
    \item Discussione su ciò che potrebbe essere fatto nel prossimo Sprint, utile come preparazione al prossimo Sprint Planning.
\end{itemize}

Durata massima: 1 ora.

\subsubsection{Sprint retrospective}
Nella Sprint retrospective viene analizzato il processo di sviluppo, in particolare si discute di:

\begin{itemize}
    \item Come durante lo Sprint passato i \textbf{meeting} sono stati affrontati e come i \textbf{tool} sono stati utilizzati;
    \item Quali possibili miglioramenti ai soggetti del punto precedente si vogliono attuare nello Sprint successivo.
\end{itemize}

Durata massima: 45 minuti.

\subsection{Divisione dei Task}

La suddivisione dei Task è effettuata inizialmente durante lo \textbf{Sprint planning}, ma durante un singolo Sprint questi possono cambiare, quindi durante un \textbf{Daily Scrum} qualora sia necessario vengono definiti nuovi task, modificati o eliminati quelli correnti, e di conseguenza l'assegnazione di questi può cambiare.

\subsection{Revisione dei Task}

Ogni task, sia durante il suo sviluppo, sia una volta portato a termine, subisce diversi controlli mirati a massimizzare la qualità del risultato.
%
Il team effettua spesso revisioni del lavoro dei membri in diverse modalità, che vengono scelte di volta in volta, a seconda delle specifiche esigenze del task in esame.

\paragraph{Code review}

Una volta terminato lo sviluppo del codice relativo a un task, quest'ultimo non viene considerato completato fino a che non passa attraverso una fase di code review.
%
Il codice prodotto dal responsabile del task viene mostrato ai restanti membri del team per individuare possibili refactor atti a migliorare la qualità del codice.
%
Generalmente si auspica che eventuali errori di programmazione vengano individuati dai tool di testing e di continuous integration.

\paragraph{Daily scrum}

Il Daily scrum costituisce uno dei momenti in cui il team valuta la bontà delle soluzioni proposte dai membri e cerca delle soluzioni per gli eventuali problemi riscontrati.
%
Sulla base di questi risultati decide quali devono essere i prossimi task da completare e le loro priorità reciproche.

\paragraph{Pair programming}

Per task di particolare delicatezza, il team non esclude la possibilità di lavorare contemporaneamente in gruppi di due o più persone sulle medesime funzionalità col fine di minimizzare errori o debiti tecnici.
%
Questo è vero specialmente nelle fasi iniziali del progetto, in cui le scelte di design e di implementazione sono ancora incerte ed è necessario allineare la conoscenza di tutti i membri riguardo il risultato atteso.

\subsection{Tool}

A supporto del processo agile, il team si impone di utilizzare strumenti di automazione con lo scopo di migliorare l'efficienza e di consentire al gruppo di concentrarsi maggiormente sullo sviluppo in sè.

In particolare vengono utilizzati i seguenti tool:
%
\begin{itemize}
    \item \textbf{SBT} come strumento di build automation;
    \item \textbf{Scalatest} per la scrittura ed esecuzione dei test automatizzati, in combinazione con \textbf{scalamock} per semplificare la creazione dei \textit{test double};
    \item \textbf{Travis CI} come strumento per la \textit{continuous integration}.
\end{itemize}
