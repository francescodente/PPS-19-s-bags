\section{Processo di sviluppo}

% (modalità di divisione in itinere dei task, meeting/interazioni pianificate, modalità di revisione in itinere dei task, scelta degli strumenti di test/build/continuous integration)

Il processo di sviluppo adottato dal team è \textbf{incrementale} ed \textbf{iterativo}, fortemente ispirato a \textbf{Scrum} ma adattato alle esigenze di un progetto dove gli sviluppatori devono ricoprire anche i ruoli di \textbf{Scrum Master} e \textbf{Product Owner}.
%
Il team effettua \textbf{Sprint} settimanali in cui decide gli obiettivi, suddivide i compiti e svolge incontri quotidiani.
%
Alla fine viene discusso il lavoro portato a termine in vista dello Sprint successivo.

\subsection{Meeting}
Gli incontri sono un tassello fondamentale per le metodologie \textbf{agili}, ed il nostro processo riprende quelle di Scrum, con tutto il team che partecipa ad ogni incontro, svolgendo sempre il ruolo di sviluppatori e dove necessario quelli di Scrum Master e Product Owner.

\subsubsection{Sprint planning}
Questo incontro è svolto a cadenza settimanale all'inizio di ogni Sprint, e vengono discussi i seguenti punti:

\begin{itemize}
    \item Identificazione dell'\textbf{incremento} che si vuole ottenere alla fine dello Sprint tramite utilizzo e rifinimento del \textbf{Product Backlog};
    \item Definizione dei \textbf{Task} come unità di lavoro pratica per soddisfare i \textbf{requisiti};
    \item Assegnazione dei Task ai membri del team.
\end{itemize}

Lo Sprint Planning ha una durata massima di 2 ore.

\subsubsection{Daily Scrum}
Durante il Daily Scrum ogni sviluppatore espone al team i seguenti punti:

\begin{itemize}
    \item Quale lavoro ha svolto la giornata precedente;
    \item Quale lavoro intende svolgere nella giornata corrente;
    \item Eventuali possibili impedimenti per il lavoro da svolgere, e come gli altri membri del team potrebbero aiutare ad affrontare il problema.
\end{itemize}

La durata di questo incontro è al massimo di 15 minuti.

\subsubsection{Sprint review}
Lo Sprint review analizza il prodotto sviluppato con i seguenti punti:

\begin{itemize}
    \item Ispezione dell'incremento ottenuto, ovvero aggiornamento del team sullo stato di sviluppo complessivo del progetto;
    \item Adattamento Product Backlog;
    \item Valutazione di eventuali ritardi sulla tabella di marcia;
    \item Discussione su ciò che potrebbe essere fatto nel prossimo Sprint, utile come preparazione al prossimo Sprint Planning.
\end{itemize}

Durata massima: 1 ora.

\subsubsection{Sprint retrospective}
Nella Sprint retrospective viene analizzato il processo di sviluppo, in particolare si discute di:

\begin{itemize}
    \item Come durante lo Sprint passato i \textbf{meeting} sono stati affrontati e come i \textbf{tool} sono stati utilizzati;
    \item Quali possibili miglioramenti ai soggetti del punto precedente si vogliono attuare nello Sprint successivo.
\end{itemize}

Durata massima: 45 minuti.

\subsection{Divisione dei Task}

La suddivisione dei Task è effettuata inizialmente durante lo \textbf{Sprint planning}, ma durante un singolo Sprint questi possono cambiare, quindi durante un \textbf{Daily Scrum} qualora sia necessario  vengono definiti nuovi task, modificati o eliminati quelli correnti, e di conseguenza l'assegnazione di questi può cambiare.

\subsection{Revisione dei Task}

\paragraph{Code review}

\paragraph{Daily scrum}

\paragraph{Pair programming}

\subsection{Tool}

\subsubsection{SBT}

\subsubsection{Scalatest con Scalamock} %Potrebbe essere meglio citare scalamock solo all'interno della sottosezione anziché nel suo titolo

\subsubsection{Travis}