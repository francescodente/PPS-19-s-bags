\section{Requisiti}

%questa sezione deve essere sufficiente per uno sviluppatore per giungere ad un sistema che fa quello che fa il nostro

%Attenzione in particolare ai requirement non funzionali: 1) non siano troppo vaghi altrimenti sono inverificabili, e quindi praticamente inutili; 2) se il sistema è distribuito, è inevitable dire cosa vi aspettate in termini di di robustezza a cambiamenti/guasti (quali?, come?), e scalabilità (in quale dimensione? fino a che punto?).

%------------------------------------------------------------------------------------

\subsection{Business}

La libreria consentirà di realizzare giochi da tavolo con le seguenti caratteristiche:

\begin{itemize}
    \item Giochi basati sul movimento di \textit{pawn} all'interno di \textit{board};
    \item Giochi a informazione perfetta; % TODO add foot note https://it.wikipedia.org/wiki/Gioco_a_informazione_completa
    % TODO
\end{itemize}

%------------------------------------------------------------------------------------
\subsection{Utente}

Per uno sviluppatore che utilizzerà la libreria sarà necessario specificare:

\begin{itemize}
    \item le componenti del gioco e le loro caratteristiche;
    \item le regole del gioco:
    \begin{itemize}
        \item insieme delle \textit{move} disponibili;
        \item condizioni che determinano la validità di una \textit{move};
        \item conseguenze dell'esecuzione di una \textit{move} sul \textit{game state};
    \end{itemize}
    \item lo svolgimento del gioco, tra cui:
    \begin{itemize}
        \item lo stato iniziale;
        \item le condizioni di avanzamento;
        \item le condizioni di terminazione;
    \end{itemize}
    \item la modalità con cui l'input dell'utente finale viene convertito in \textit{move}. % TODO da sistemare
\end{itemize}

\subsection{Funzionali}

% TODO intro della sezione
La libreria implementerà gli aspetti principali della gestione di giochi da tavolo e dovrà offrire agli sviluppatori un'interfaccia che permetterà agli sviluppatori di:

\begin{itemize}
    \item definire gli aspetti statici del gioco, ovvero i dati che caratterizzano \textit{player}, \textit{tile}, \textit{pawn}, \textit{board};
    \item definire gli aspetti dinamici del gioco, ovvero le regole, le \textit{move} e i loro effetti sul gioco;
\end{itemize}

\subsection{Non funzionali}

% TODO intro della sezione


\subsection{Di implementazione}

% TODO intro della sezione