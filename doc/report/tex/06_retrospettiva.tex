\section{Restrospettiva}

%(descrizione finale dettagliata dell'andamento dello sviluppo, del backlog, delle iterazioni; commenti finali)

%Si noti che la retrospettiva è l'unica sezione che può citare aneddoti di cosa è successo in itinere, mentre le altre sezioni fotografino il risultato finale. Se gli studenti decideranno (come auspicato) di utilizzare un product backlog e/o dei backlog delle varie iterazioni/sprint, è opportuno che questi siano file testuali tenuti in versione in una cartella "process", così che sia ri-verificabile a posteriori la storia del progetto.

\subsection{Sprint 1}
Il primo sprint è stato principalmente incentrato sul setup del progetto, l'analisi del problema e la creazione delle interfacce core della libreria, questo ha portato a meeting molto frequenti e prolungati durante la giornata.
\paragraph{Svolgimento e sviluppo dello sprint}
Sono stati inizializzati i tool inerenti al processo di sviluppo, in particolare:
\begin{itemize}
   \item \textbf{SBT} per la gestione delle dipendenze e delle build;
   \item \textbf{Travis CI} per il testing automatico e la Continous Integration.
\end{itemize}
Si è ritenuto necessario sin da subito creare convenzioni comuni riguardo i termini, la documentazione e la scrittura di codice, ad esempio l'uso di un glossario comune.
%
Dopo una prima parte di Design, sono stati scritti i test e le relative interfacce necessarie alla creazione del gioco PutInPutOut; infine sono stati creati i test del gioco tramite i requisiti specificati ed è stato implementato il gioco.
\paragraph{Considerazioni finali}
Lo sprint non ha sollevato particolari problemi ed è terminato nei tempi previsti.

Una possibile prosecuzione prevede:
\begin{itemize}
  \item la possibilità di avere dei \textbf{turni} e dei giocatori all'interno della partita;
  \item un'interfaccia più facilmente utilizzabile per l'utente della libreria;
  \item l'implementazione del gioco \textbf{Tic-Tac-Toe}.
\end{itemize}

\subsection{Sprint 2}
Il secondo sprint è stato principalmente incentrato sull'estensione delle funzionalità core della libreria, lo sviluppo di utility comunemente adottate dai giochi target della libreria e la creazione del gioco Tic-Tac-Toe.
\paragraph{Svolgimento e sviluppo dello sprint}
Lo sviluppo della libreria è stato parallelizzato alla creazione di Tic-Tac-Toe.
%
Alcuni dei requisiti di Tic-Tac-Toe sono stati resi più generali al fine di integrarli nella libreria: in particolare la necessità di avere dei \textit{player}, un \textit{game flow} basato su turni e su \textit{game ending conditions}.
\paragraph{Considerazioni finali}
Durante lo sprint si sono riscontrati problemi con l'utilizzo di \textbf{ScalaMock} e, data la complessità del progetto, il team si è trovato a lavorare sulle stesse funzionalità, portando a una minore parallelizzazione dei task e, di conseguenza, a un minor rigore nel processo di sviluppo.

Una possibile prosecuzione prevede:
\begin{itemize}
  \item un'interfaccia più facilmente utilizzabile per l'utente della libreria;
  \item l'implementazione di una interfaccia visuale del gioco \textbf{Tic-Tac-Toe}.
\end{itemize}

\subsection{Sprint 3}
Il terzo sprint aveva inizialmente lo scopo di sviluppare un primo DSL ed un'interfaccia utente testuale, ma a causa di rigidità del codice precedente è stato necessario operare un \textbf{refactor} su una grande porzione di codice, e come risultato non è stato possibile raggiungere gli obiettivi prefissati.
\paragraph{Considerazioni finali}
Dal punto di vista del processo di sviluppo la metodologia \textbf{Scrum} non è stata seguita in toto, in particolare i Daily Scrum non sempre hanno avuto luogo seguendo gli step prestabiliti, ma si è proceduto in gruppo a lavorare sul codice da rifattorizzare.

Una possibile prosecuzione prevede il completamento dei Task ancora aperti a cui va ad aggiungersi lo sviluppo del gioco \textbf{Connect Four}.

\subsection{Sprint 4}
Il quarto Sprint ha raggiunto con successo gli obiettivi posti: lo sviluppo di un primo DSL, di un'interfaccia utente testuale, e del gioco \textbf{Connect Four}.
\paragraph{Considerazioni finali}
La metodologia \textbf{Scrum} è stata applicata efficacemente, correggendo gli errori fatti negli Sprint precedenti.
%
Una possibile miglioria riguardante gli strumenti a supporto del processo di sviluppo è quella di individuare dei \textbf{Task} a grana più fine ed inserirli in \textbf{Trello}, al fine di guadagnare in autonomia dei membri del team, e di ottenere una maggiore chiarezza su quali funzionalità siano già pronte, quali in sviluppo, e quali totalmente assenti.

Una possibile prosecuzione prevede di migliorarela qualità del codice, ed aggiungere alcune funzionalità in più per quanto riguarda il DSL e l'interfaccia testuale.
