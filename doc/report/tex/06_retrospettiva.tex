\section{Restrospettiva}

%(descrizione finale dettagliata dell'andamento dello sviluppo, del backlog, delle iterazioni; commenti finali)

%Si noti che la retrospettiva è l'unica sezione che può citare aneddoti di cosa è successo in itinere, mentre le altre sezioni fotografino il risultato finale. Se gli studenti decideranno (come auspicato) di utilizzare un product backlog e/o dei backlog delle varie iterazioni/sprint, è opportuno che questi siano file testuali tenuti in versione in una cartella "process", così che sia ri-verificabile a posteriori la storia del progetto.

\subsection{Sprint 1}

