\section{Implementazione}

% questa sezione deve essere sufficiente per uno sviluppatore per avere una implementazione, essenzialmente equivalente alla nostra

% (per ogni studente, una sotto-sezione descrittiva di cosa fatto/co-fatto e con chi, e descrizione di aspetti implementativi importanti non già presenti nel design)


% L'implementazione "esplode" il design, ma solo laddove pensiate che serva dire qualcosa.

% Cercate di dare una idea di quanto pensate che i vostri test automatizzati coprano il codice e dove: è importante per stimare il potenziale impatto di una modifica al software.

%Scaletta:
%Setup ed esecuzione dei giochi (Program lifecycle su trello)

% - Cli
% Generazione automatica
% personalizzazione grafica

%DSL

%Estensioni
\subsection{Estensioni}
%
Ogni \textit{Game} può presentare diverse caratteristiche, quindi è necessario che le \texttt{estensioni} siano opzionali e facili da definire.
%

%Example (giochi)

%Divisione del lavoro
\subsection{Divisione del lavoro}
%Testing
\subsection{Testing}
I test effettuati sul codice sviluppato hanno lo scopo principalmente di garantire la qualità del codice, di favorire il cambiamento ed infine di documentazione del software sviluppato, seguendo la \textbf{quality school} e la \textbf{agile school} come filosofie di riferimento.
%
Avendo approcciato il progetto con la metodologia \textbf{TDD} la maggior parte del codice risulta avere degli unit test che coprono le singole funzionalità.
%
Ci sono alcune eccezioni, ad esempio l'interfaccia testuale risulta essere poco coperta dai test a causa della necessità di acquisire input da tastiera e la scarsa utilità di testare i risultati di stampe a video.
%
La copertura risulta invece molto alta nel \texttt{model}, arrivando ad avere il 100\% di coverage per il \texttt{core}.
%
Oltre agli \textbf{unit test} utilizzati per l'approccio TDD e sviluppati prima del codice stesso, sono presenti anche degli \textbf{integration test}, aggiunti una volta terminato lo sviluppo del codice di una singola unità per assicurare la corretta interazione con le altre.
%
Infine, i \textbf{system test} sono stati effettuati nei giochi d'esempio, che forniscono un ambiente articolato e completo dove poter testare l'interazione fra i diversi moduli del sistema.
%

\subsubsection{Test doubles}
Ove possibile sono stati effettuati test di tipo funzionale e \textbf{blackboxed}, mentre dove è risultato necessario sono stati effettuati dei test strutturali \textbf{whiteboxed}.
%
Entrambi le casistiche hanno visto un impiego frequente dei \textbf{test doubles} sfruttando la libreria \textbf{scalamock} per rimuovere le dipendenze dagli unit test o per verificare il comportamento interno di un componente nel caso di test whitebox.

\subsubsection{Stile dei test}
Lo stile adottato è stato \textbf{FlatSpec} nella quasi totalità dei casi, in quanto il più adatto agli unit test e semplice sia da consultare che da modificare, a favore di uno sviluppo agile del software a fronte di cambiamenti nei requisiti o nella loro comprensione da parte del team.