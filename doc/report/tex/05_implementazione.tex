\section{Implementazione}

La seguente sezione va ad analizzare quelli che sono gli aspetti e le decisioni implementative che caratterizzano il codice.
%
Si riportano solo i casi degni di nota e si lascia il resto al codice e alla documentazione su di esso, che sono stati pensati per essere autoesplicativi e semplici da comprendere per un utilizzatore della libreria.

% ---------------------------------------------------

\subsection{Passaggio implicito delle type class}

Come anticipato nella Sezione \ref{sec:detailed_design}, il meccanismo delle type class è stato sfruttato molto frequentemente.
%
Al fine di massimizzare la pulizia e la leggibilità del codice, si è deciso di utilizzare il passaggio implicito dei parametri per evitare di dover comunicare esplicitamente le istanze delle type class.

% ---------------------------------------------------

\subsection{Sintassi del DSL}

Il DSL è stato scritto in maniera quanto più auto-esplicativa possibile, in modo da mettere l'utente della libreria nella condizione di dover fare il minor sforzo possibile per la scrittura delle regole di un gioco.
%
Questo ci ha portato alla scrittura di un DSL molto simile (ove possibile) al linguaggio naturale e che richiede una conoscenza estremamente limitata del linguaggio Scala.

Per ottenere questo risultato è stato necessario sfruttare alcune funzionalità offerte da Scala:
\begin{itemize}
  \item \textbf{metodi come operatori infissi}, che richiedono quindi che un metodo sia preceduto e seguito da un oggetto;
  \item \textbf{conversioni implicite}, per aggiungere metodi a tipi già esistenti o per adattare oggetti in maniera trasparente;
  \item \textbf{uso delle parentesi graffe al posto delle parentesi tonde}, per avere una strutturazione più familiare nel caso di nesting dei costrutti.
\end{itemize}

I meccanismi citati, specialmente il primo, richiedono spesso l'adattamento della struttra sintattica delle frasi, scritte in linguaggio naturale, per fare in modo che sia supportata a livello sintattico anche da Scala.
%
In particolare si è cercato di adattare il più possibile le frasi del DSL, per fare in modo che la struttura potesse essere ricondotta a sequenze del tipo \texttt{object method object method object ...}, in modo da evitare la presenza di parentesi.

Nel Listato \ref{lst:dsl_example} è mostrato uno snippet di codice che mostra a grandi linee le modalità con cui i meccanismi di scala sono stati adottati nella scrittura del costrutto iterativo.
%
\lstinputlisting[language=scala, caption={Dichiarazione della sintassi di iterazione e esempio di utilizzo}, label=lst:dsl_example]{code/dsl.scala}
%
Un'istruzione di iterazione parte dalla parola chiave \texttt{iterating}, ovvero un oggetto il cui unico metodo è \texttt{over}.
%
Questo metodo accetta una \texttt{Feature} che estrae dallo stato un \texttt{Traversable[F]} su cui deve essere eseguita l'iterazione.
%
Come anticipato nella Sezione \ref{sec:dsl_design}, è necessario usare una \texttt{Feature} poichè le regole vengono definite a livello intensionale, quindi non si ha accesso diretto allo stato.

Il metodo \texttt{over} restituisce un nuovo oggetto di tipo \texttt{Iteration[F]}, che a sua volta contiene un unico metodo \texttt{as} al quale deve essere passata la funzione di iterazione.
%
Questa è una funzione che associa ad ogni elemento estratto dallo stato un nuovo oggetto di tipo \texttt{T} generico.
%
Per fare in modo che l'iterazione sia possibile, è necessario anche che sia implicitamente definita un'istanza di \texttt{Chainable}, per stabilire in che modo concatenare le \texttt{T} restituite dalle singole iterazioni e quale deve essere il valore di partenza per l'accumulazione.

Infine viene mostrato un esempio di utilizzo del costrutto di iterazione su una proprietà \texttt{Traversable} dello stato.
%
L'esempio utilizza il metodo \texttt{generate} fornito dalla libreria, che restituisce un oggetto di tipo \texttt{Generator}, per cui la libreria fornisce già un'istanza di \texttt{Chainable}.

% ---------------------------------------------------

\subsection{Cli}
%
Data l'infrastruttura di gestione dell'interazione con il giocatore, un utente della libreria può scegliere se scrivere una propria implementazione dell'interfaccia grafica oppure se avvalersi dell'interfaccia a riga di comando fornita dalla libreria.
%
Questa opzione è compatibile solamente con i giochi aventi una \textit{Board} \textbf{rettangolare}, per tutti gli altri è necessario fornire un'implementazione alternativa.

L'interfaccia a riga di comando è generata a partire dalle caratteristiche del gioco e dai \texttt{Renderer} associati, seguendo così un approccio standardizzato per associare a diverse \textbf{estensioni} del gioco una specifica rappresentazione in maniera automatica.
%
Ciò consente di avere una visualizzazione semplice del gioco scritto usando la libreria, con il minimo sforzo da parte dell'utente, che ha però la possibilità di personalizzare la grafica fornendo una serie di parametri oppure nuove implementazioni dei singoli componenti per i quali desidera un aspetto diverso.
%
In particolare gli aspetti personalizzabili per il \texttt{BoardRenderer} sono:
\begin{itemize}
  \item Rappresentazione degli indici delle colonne;
  \item Rappresentazione degli indici delle righe;
  \item Separatore degli elementi grafici;
  \item Terminazione riga;
  \item Rappresentazione dei \textit{Tile} vuoti;
  \item Rappresentazione dei diversi \textit{Pawn}.
\end{itemize}
%
Ognuno di questi parametri ha un valore di default, seguendo il principio della \textbf{convenzione} prima della \textbf{personalizzazione} in modo da snellire il processo di preparazione dell'interfaccia da parte dell'utente qualora le impostazioni base fossero sufficienti.
%
Infine, i \texttt{Renderer} rimanenti non sono parametrici in quanto sufficientemente semplici da poter essere completamente sostituiti in maniera rapida da nuove implementazioni.

% ---------------------------------------------------

\subsection{Configurazione dell'app: GameSetup}

La configurazione degli elementi che gestiscono l'interazione tra un giocatore umano e l'applicazione finale richiede svariati passaggi.
%
È necessario che lo sviluppatore configuri correttamente i parametri di rendering, la comunicazione tra view e controller e altri dettagli spesso dipendenti l'uno con l'altro.

Per porre rimedio a questo problema e per fornire un singolo punto di configurazione, è stato creato un piccolo modulo di utility -- il \texttt{GameSetup} -- che ha l'obiettivo di aiutare lo sviluppatore fornendo uno scheletro della configurazione, implementato con valori di default considerati standard.
%
Questo modulo consente a chi utilizza la libreria di fornire la propria configurazione personalizzata facendo override dei metodi presenti nel \texttt{GameSetup} in stile \textbf{template method}.

La libreria supporta al momento una configurazione dettagliata solo per l'interfaccia testuale (estendendo il trait \texttt{CliGameSetup}), ed in particolar modo aiuta lo sviluppatore nel caso venga utilizzata una board di tipo rettangolare (aggiungendo il mixin \texttt{RectangularBoardSetup}).
%
Le principali funzionalità che sono state fornite sono:
%
\begin{itemize}
  \item Scelta dei renderer che dovranno essere utilizzati a partire da un \textbf{builder}, per poter scegliere dinamicamente cosa disegnare;
  \item Configurazione del parsing dei comandi a partire da un \textbf{builder}, per aggiungere dinamicamente comandi con sintassi arbitrarie;
  \item Sintassi semplificata per i metodi di entrambi i builder, che sono internamente implementati come template method che richiamano i metodi potenzialmente sovrascrivibili dallo sviluppatore stesso.
\end{itemize}

Una volta scritta la propria configurazione nel \texttt{GameSetup}, è possibile avviare l'applicazione chiamando \texttt{AppRunner.run} fornendo la propria istanza di \texttt{GameSetup}, che viene generalmente chiamato dall'interno dell'entry point dell'app.

% ---------------------------------------------------

\subsection{Giochi}
Oltre alla libreria stessa sono stati sviluppati alcuni giochi per fungere da esempi per gli utenti e dimostrare l'utilizzo delle funzionalità proposte.
%
I giochi realizzati sono i quattro descritti nei requisiti, e sono stati scelti sia per le loro caratteristiche che per fornire diversi livelli di complessità implementativa.

Si fa ora una breve analisi della struttura di \textit{Connect four} come esempio cardine.
%
Questo gioco è stato scelto poiché la sua complessità è tale da mostrare le potenzialità della libreria senza richiedere un'analisi troppo complicata.
%
Per scrivere un gioco si parte dalla defizionione di quelli che sono gli aspetti del modello, ovvero: 
\begin{itemize}
  \item il trait \texttt{ConnectFourPawn} che rapresenta i \textit{pawn};
  \item la \textit{move} \texttt{Put} che indica l'inserimento di una pedina all'interno di una colonna;
  \item \texttt{ConnectFourBoard} una board rettangolare;
  \item \texttt{ConnectFourState} che racchiude il riferimento alla \textit{board} e al \textit{player} corrente.
\end{itemize}
%
Definiti quelli che sono gli aspetti del modello si passa alla definizione del \textit{rule set}.
%
Questo estende da \texttt{RuleSet} ed utilizza il mixin \texttt{RuleSetBuilder} per abilitare le funzionalità del DSL offerte dalla libreria.
%

\lstinputlisting[language=scala, caption={Connect four rule set}, label=lst:connect4_rule_set_example]{code/ConnectFourRuleSet.scala}

%
Come è possibile vedere dal codice Listato \ref{lst:connect4_rule_set_example} si definisce una apposita \texttt{Feature} per estrarre dallo stato la prima \textit{tile} libera di una colonna.
%
Si usa quindi questa \texttt{Feature} appena definita per specificare in quale modo l'azione \texttt{Put} sopra definita agisca sullo stato, ovvero, (traducendo quello che in inglese è presente nel codice) piazzando il turno corrente sulla prima cella libera della colonna x.
%
Si definisce poi quando il turno deve cambiare, ovvero dopo ogni mossa.
%
Infine, si specifica che la mossa \texttt{Put} è valida e deve essere generata per ogni colonna in cui la \textit{tile} che si trova in cima non è occupata.

Ora è necessario creare una entità che estenda da \texttt{GameDescription} dentro la quale definire tutte le caratteristiche che deve avere il gioco.
%
\lstinputlisting[language=scala, caption={Connect four game description}, label=lst:connect4_game_description]{code/ConnectFourGDescr.scala}
%
In particolare, si definiscono:
\begin{itemize}
  \item lo stato iniziale;
  \item il \textit{rule set};
  \item i player;
  \item i turni associandoli ai player, in questo modo \texttt{currentTurn} rapresenta sia il turno che il player;
  \item le \textit{endCondition} che definiscono le condizioni di terminazione di una partita.
\end{itemize}
%
Si nota, nel Listato \ref{lst:connect4_game_description}, che i valori sono definiti in modo implicito per semplificare l'utilizzo del \textit{rule set} e del resto della libreria.

\lstinputlisting[language=scala, caption={Connect four main e game setUp}, label=lst:connect4_main]{code/ConnectFourMain.scala}
%
Infine, come mostrato nel Listato \ref{lst:connect4_main}, definendo un \texttt{ConnectFourSetUp} che estende da \texttt{CliGameSetUp} si ottiene senza alcuno sforzo una rappresentazione testuale del gioco e la possibilità di eseguire il gioco appena scritto.

% ---------------------------------------------------

\subsection{Testing}
I test effettuati sul codice sviluppato hanno lo scopo principalmente di garantire la qualità del codice, di favorire il cambiamento ed infine di documentazione del software sviluppato, seguendo la \textbf{quality school} e la \textbf{agile school} come filosofie di riferimento.
%
Avendo approcciato il progetto con la metodologia \textbf{TDD} la maggior parte del codice risulta avere degli unit test che coprono le singole funzionalità.
%
Ci sono alcune eccezioni, ad esempio l'interfaccia testuale risulta essere poco coperta dai test a causa della necessità di acquisire input da tastiera e la scarsa utilità di testare i risultati di stampe a video.
%
La copertura risulta invece molto alta nel \texttt{model}, arrivando ad avere il 100\% di coverage per il \texttt{core}.
%
Oltre agli \textbf{unit test} utilizzati per l'approccio TDD e sviluppati prima del codice stesso, sono presenti anche degli \textbf{integration test}, aggiunti una volta terminato lo sviluppo del codice di una singola unità per assicurare la corretta interazione con le altre.
%
Infine, i \textbf{system test} sono stati effettuati nei giochi d'esempio, che forniscono un ambiente articolato e completo dove poter testare l'interazione fra i diversi moduli del sistema.

\subsubsection{Test doubles}
Ove possibile sono stati effettuati test di tipo funzionale e \textbf{blackboxed}, mentre dove è risultato necessario sono stati effettuati dei test strutturali \textbf{whiteboxed}.
%
Entrambi le casistiche hanno visto un impiego frequente dei \textbf{test doubles} sfruttando la libreria \textbf{scalamock} per rimuovere le dipendenze dagli unit test o per verificare il comportamento interno di un componente nel caso di test whitebox.

\subsubsection{Stile dei test}
Lo stile adottato è stato \textbf{FlatSpec} nella quasi totalità dei casi, in quanto il più adatto agli unit test e semplice sia da consultare che da modificare, a favore di uno sviluppo agile del software a fronte di cambiamenti nei requisiti o nella loro comprensione da parte del team.

% ---------------------------------------------------

\subsection{Divisione del lavoro}
%parti comuni:
% - core
% - extension
% - divisione in due sottogruppi per la gestione delle due macro-componenti del progetto
% -- Comuni: RuleSet, DSL, Features
La divisione del lavoro del team si può suddividere in tre stadi:
%
Inizialmente il gruppo ha affrontato l'analisi del dominio e l'analisi architetturale dell'intero sistema in completa collaborazione, tramite meeting intensi nei quali si è cercato di delineare quelli fossero gli obiettivi del team nel breve e lungo periodo.
%
Una volta stabilito quello che è il design architetturarale del sistema si è scelto di dividere il team in due sotto-team di due persone ciascuno.
%
Il primo gruppo è quello formato da Dente ed Evangelisti al quale è stato assegnato il compito di lavorare sul \textit{rule set} e sul DSL relativo.
%
Il secondo gruppo formato da Magnani e Nemati si è occupato principalemente della parte di interaction.
%
In fine, all'interno di ogni gruppo, dopo aver effettuato un'attenta analisi in dettaglio delle sezioni di interesse, si è suddiviso il lavoro per singoli membri.

Le divisioni del lavoro svolte sono state sempre ben documentate nel corso dello svolgimento del progetto ma è da tenere in considerazione che la stretta collaborazione e il continuo ciclo di daily scrum ha fatto in modo che tutti i membri del gruppo fossero sempre tenuti al corrente dell'andamento del progetto.
%
Si sottolinea che buona parte del codice è stata scritta in \textbf{pair programming} per massimizzare la qualità del codice e ridurre il tempo relativo alla revisione del codice.

\subsubsection{Dente Francesco}

La porzione principale del progetto di cui mi sono occupato è stata la scrittura e la progettazione del DSL per la definizione del \textit{rule set}, svolta in collaborazione con Evangelisti.
%
In particolare, nonostante gran parte del lavoro sia stata portata avanti cooperativamente, ho dato un contributo particolare nella scrittura del codice relativo alla generazione delle mosse e all'utilizzo del concetto di Chainable all'interno dei costrutti iterativi e condizionali.

Per quanto riguarda invece la parte core della libreria, mi sono occupato dell'introduzione delle type class per implementare le estensioni dello stato, nonostante queste siano state poi aggiornate con le successive iterazioni in maniera cooperativa da tutti i membri del gruppo.

Un'altra area su cui mi sono concentrato è stata l'implementazione delle astrazioni di GameSetup e in generale della gestione della fase di configurazione e avvio dell'applicazione.

Infine, mi sono occupato dello sviluppo e del testing del gioco d'esempio Othello, che è stato implementato una volta terminata la scrittura del codice della libreria in sè.

\subsubsection{Evangelisti Davide}
% dsl in comune con Dente, RuleSet (con particolare attenzione alla move execution e le Action), PutInPutOut Revisited, 'after each move', RuleSetBuilder
% RuleSetBuilder
% MovesExecution
Mi sono occupato principalemente della scrittura del \textit{rule set} assieme al collega Dente con il quale ho affrontato l'analisi del problema e la progettazione di massima del DSL.
%
La parte, del \textit{rule set} alla quale mi sono dedicato maggiormente è stata quella relativa all'esecuzione delle \textit{move} e al relativo DSL, che permette all'utente della libreria di scrivere facilmente in quale modo lo stato del gioco viene modificato da un'azione.
%
Nello specifico mi sono occupato del trait \texttt{MovesExecution}, nel quale, tramite le funzionalità offerte da Scala, ho costruito la parte di DSL relativa all'esecuzione delle \textit{move}.
%
Grazie alle conversioni implicite sono riuscito a rendere la \texttt{MovesExecution} indipendente dal sistema delle \texttt{Actions} e allo stesso tempo queste ultime si integrano perfettamente all'interno del sistema di creazione delle \textit{move}.

Per quanto riguarda il \texttt{RuleSet} e il \texttt{RuleSetBuiledr} ho scelto di dividere questi due aspetti in modo da rendere il \texttt{RuleSet} indipendente dal modo in cui questo è generato.
%
Così facendo do la possibilità a chi utilizza la libreria di scegliere se utilizzare il DSL oppure approciarsi alla generazione e all'esecuzione delle mosse in un modo diverso.

Oltre alla parte relativa al \textit{rule set} mi sono occupato anche della \textit{Board} e della sua scrittura.
%
In questo particolare frangente non sono presenti tecniche di programmazione avanzate se non l'immutabilità della \textit{Board} stessa.
%
Si è scelto di mantenere la \textit{Board} immutabile in vista di modifiche future e di funzionalità aggiuntive della libreria, questa scelta si è dimostrata corretta in quanto durante la scrittura della funzionalità opzionale di \textbf{undo}, ovvero la possibilità di tornare allo stato precedente, la presenza di un oggetto immutabile semplifica enormemente il lavoro.

Oltre a questo mi sono occupato dello sviluppo e del mantenimento del primo gioco sviluppato dal gruppo, ovvero, PutInPutOut.

\subsubsection{Magnani Simone}
% interaction in comune con Nemati. In singolo:
Dopo lo sviluppo delle parti comuni e la divisione in sottogruppi, insieme a Nemati, abbiamo lavorato allo sviluppo di tutto ciò che riguarda l'interazione con l'utente.
% View - GameView
In particolare, occupandomi della \texttt{View} ho cercato di rendere la generazione di ogni componente il più lontano possibile dal \texttt{model}, tanto che per giochi con \textit{Board} rettangolari esiste un'implementazione automatica della visualizzazione ed interazione.
% renderer
Ho sviluppato i \texttt{Renderer} in modo da comporre la \texttt{GameView}, ma incapsulando l'obiettivo specifico di ogni \texttt{Renderer}.
%
Questo ha reso la \texttt{GameView} indipendente dalle varie \texttt{extension} di ogni particolare gioco permettendo la generazione automatica della stessa.
%
\'E importante anche menzionare il \texttt{CliBoardRenderer}, che tramite la configurazione dei \texttt{Converter} nel \texttt{CliGameSetup}, rende personalizzabile la visualizzazione della \textit{Board} in relazione alla selezione dei \textit{Tile}.
% Connect4
Infine mi sono occupato del testing e dell'implementazione di \textbf{Connect Four}, il che mi ha portato a realizzare delle \texttt{apply} di utility per le \texttt{extension}.

\subsubsection{Nemati Shapour}
% interaction in comune con Magnani.
In seguito al lavoro comune iniziale, la parte a cui ho contribuito maggiormente è stata quella di interazione utente, assieme a Magnani.
%In singolo:
% Controller
Personalmente ho lavorato al \textbf{controller}, in particolare al \textbf{GameController} ed alla sua interazione con il \textbf{Model}, gestendo l'unico punto di contatto fra questi, anche se successivamente la parte di gestione degli errori è stata spostata dal controller al model.
% Event
Per gestire l'interazione fra \textbf{controller} e \textbf{view} ho gestito lo sviluppo degli \texttt{Event} come elemento base di interazione che possa essere indipendente dal tipo di view e sempre costante per il controller.
% Input parser
Nel lavorare a stretto contatto con la parte di \textbf{cli} ho poi sviluppato l'\textbf{Input Parser} che gestisce la conversione dal testo digitato dall'utente agli \texttt{Event} gestibili dal controller, anche se successivamente questa parte è stata riadattata da Dente per semplificarne l'utilizzo nel caso in cui lo sviluppatore non volesse ricorrere alle espressioni regolari per definire il proprio mapping fra stringhe ed eventi.

% Rectangular Board
Per quanto riguarda il mio contributo ai giochi ho espanso le funzionalità delle \textit{board} rettangolari in modo da fornirne le diagonali all'utente, che in svariati giochi potrebbe voler conoscere lo stato di una o più diagonali.

% Tic-Tac-Toe
Infine ho sviluppato il gioco Tic-Tac-Toe, utile campo di prova per la generazione automatica della \textbf{cli} oltre che delle estensioni.