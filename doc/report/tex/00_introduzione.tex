\section*{Introduzione}
Questo report è relativo allo sviluppo della libreria SBAGS e dei seguenti giochi: TicTacToe, Connect Four ed Othello.
%
La libreria permette di realizzare giochi di strategia astratti basati su scacchiera, descrivendo alcuni aspetti statici e strutturali tramite i costrutti propri del linguaggio Scala, e gli aspetti più dinamici tramite un DSL in un linguaggio simil-naturale.
%
SBAGS ha come target principale giochi in cui due o più giocatori interagiscono a turni alterni, spostando pedine sulle tessere di una scacchiera e seguendo un insieme di regole ben definite fino a quando non viene raggiunto uno stato terminale.
%
Alcuni esempi di questi giochi sono Scacchi, Dama, TicTacToe, Connect Four ed Othello.

Se un utente desidera realizzare un gioco con caratteristiche convenzionali dovrà svolgere uno sforzo minimo, infatti, per il caso di board rettanglare viene generata automaticamente dalla libreria un'interfaccia grafica testuale.
%
Qualora il gioco da realizzare dovesse necessitare di feature non presenti nella libreria l'utente potrà facilmente espanderla aggiungendo i moduli necessari, guidato dalla struttura della libreria.
%
Una volta fatto questo, le nuove funzionalità saranno integrate a quelle già presenti, ottenendo così la possibilità di godere delle caratteristiche di entrambi.

\addcontentsline{toc}{section}{Introduzione}
