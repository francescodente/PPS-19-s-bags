\section{guide utente}
%
In questa sezione conclusiva si mostra come la libreria può essere usata e quelli che sono i casi notevoli e le feature più importati sviluppate.
%
I paragrafi seguenti mostrano, seguendo l'ordine consigliato, i passi per la creazione di un gioco utilizzando SBAGS, prendendo come esempio il gioco \textit{TicTacToe}.

\paragraph{Pawn}
%
È utile definire un \texttt{trait} che rappresenti i \textit{pawn} in maniera generale e definire dei \texttt{case object} o delle \texttt{case class} che rappresentino le pedine del gioco come ad esempio nel Listato \ref{lst:pawn_def}.
%
\lstinputlisting[language=scala, caption={Definizione di pawn}, label=lst:pawn_def]{code/guide/PawnExample.scala}

\paragraph{Tile}
%
Come per i pawn, è necessario definire il proprio tipo di \textit{tile} da utilizzare poi nella definizione della board structure.
%
Nel caso di \textit{board rettangolare} è possibile utilizzare il tipo \texttt{Coordinate}, già definito dalla libreria come da Listato \ref{lst:coordinate}.
%
\lstinputlisting[language=scala, caption={Definizione di tile}, label=lst:coordinate]{code/guide/Coordinate.scala}

\paragraph{BoardStructure}
%
Per la definizione della \textit{board} è possibile utilizzare \texttt{Rec\-tan\-gu\-lar\-Struc\-ture} per definire una board rettangolare, nel Listato \ref{lst:board_def} che viene mostrata una board valida per il gioco \textit{TicTacToe}.
%
\lstinputlisting[language=scala, caption={Definizione della board}, label=lst:board_def]{code/guide/BoardExample.scala}

\paragraph{Move}
Anche per le \textit{move} l'utente deve definire un proprio tipo base.
%
Solitamente deve essere definito un \texttt{trait} in cima alla gerarchia e un certo numero di \texttt{case class}/\texttt{case object} che estendono da esso.
%
Per \textit{TicTacToe} sono definite come mostrato nel Listato \ref{lst:move}.
%
\lstinputlisting[language=scala, caption={Definizione delle move}, label=lst:move]{code/guide/Moves.scala}

\paragraph{State}
%
La definizione dello state avviene solitamente utilizzando una \texttt{case class} che contiene tutto ciò che è necessario al \textit{rule set} per funzionare.
%
Nel caso di \textit{TicTacToe} lo stato è definito come nel Listato \ref{lst:state}.
%
\lstinputlisting[language=scala, caption={Definizione dello state}, label=lst:state]{code/guide/State.scala}

\paragraph{GameDescription}
%
Una volta definito il modello è possibile iniziare a definire gli aspetti più dinamici del gioco.
%
Il primo di questi è la \texttt{GameDescription}, che deve fornire lo stato iniziale della partita e il \texttt{RuleSet} che utilizza.
%
Non essendo ancora stato implementato, quest'ultimo verrà aggiunto in una fase successiva.
%
La \texttt{GameDescription} per \textit{TicTacToe} è inizialmente implementata come da Listato \ref{lst:game_descr}.
%
\lstinputlisting[language=scala, caption={Definizione della GameDescription}, label=lst:game_descr]{code/guide/GameDescription.scala}

\paragraph{Extension}
%
Si mostra ora come è utile utilizzare le \texttt{extension}, nel Listato \ref{lst:extension}.
%
\lstinputlisting[language=scala, caption={Uso delle estensioni}, label=lst:extension]{code/guide/Extension.scala}
%
TicTacToe dichiara come estensioni, oltre allo stato della \textit{board}, l'utilizzo di turni corrispondenti ai giocatori (\texttt{PlayersAsTurns}) e una condizione di vittoria o pareggio (\texttt{WinOrDrawCondition}).

\paragraph{RuleSet}
Il passo successivo dopo la definizione delle estensioni è la definizione del \texttt{RuleSet}.
%
Per farlo è necessario definire le regole per l'esecuzione delle mosse e per la loro generazione.

Il Listato \ref{lst:rule_set0_def} mostra come sia possibile assegnare alla mossa \texttt{Put(t)} l'azione che piazza il simbolo del giocatore attivo (che è del tipo \texttt{TicTacToePawn}) sulla \textit{tile} \texttt{t}.
%
\lstinputlisting[language=scala, caption={Definizione del rule set tramite DSL - 1}, label=lst:rule_set0_def]{code/guide/RuleSetExample0.scala}
%
In seguito si specifica che dopo ogni mossa il turno deve cambiare nel Listato \ref{lst:rule_set1_def}.
\lstinputlisting[language=scala, caption={Definizione del rule set tramite DSL - 2}, label=lst:rule_set1_def]{code/guide/RuleSetExample1.scala}
%
Infine, si specifica quando la mossa \texttt{Put} debba essere creata, ovvero per ogni \textit{tile} vuota, nel Listato \ref{lst:rule_set2_def}.
%
\lstinputlisting[language=scala, caption={Definizione del rule set tramite DSL - 3}, label=lst:rule_set2_def]{code/guide/RuleSetExample2.scala}

La fase successiva prevede di definire la \texttt{GameDescription} come \texttt{object} che estende \texttt{GameDescription}, dove sono dichiarati gli \texttt{implicit} necessari al funzionamento delle \texttt{extension} che si desidera utilizzare.

\paragraph{GameSetup e Main}
Una volta definito il gioco, per creare l'applicazione finale ci si può avvalere del \texttt{GameSetup}, o più in particolare del \texttt{CliGameSetup} se si vuole usufruire della cli già fornita, dove vengono definiti i \texttt{Renderer} ed i parametri del \textbf{controller} (Listato \ref{lst:game_setup}).
%
\lstinputlisting[language=scala, caption={Definizione del GameSetup}, label=lst:game_setup]{code/guide/GameSetup.scala}
%
Infine è necessario sviluppare un \textbf{main} che sia in grado di eseguire l'applicazione così definita utilizzando l'\texttt{AppRunner} come nel Listato \ref{lst:main_app}.
%
\lstinputlisting[language=scala, caption={Definizione del main}, label=lst:main_app]{code/guide/Main.scala}