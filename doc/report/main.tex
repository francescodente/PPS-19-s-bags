\documentclass{scrartcl}
\usepackage[utf8]{inputenc}
\usepackage{hyperref}
\usepackage{url}
\usepackage{natbib}
\usepackage{graphicx}

\newcommand{\emailaddr}[1]{\href{mailto:#1}{\texttt{#1}}}

\title{\LARGE
    PPS Final Report
}

\author{
    Davide Evangelisti \\ \emailaddr{davide.evangelisti2@studio.unibo.it}
    \and 
    Francesco Dente \\ \emailaddr{francesco.dente@studio.unibo.it} 
    \and 
    Shapour Nemati \\ \emailaddr{shapour.nemati@studio.unibo.it}
    \and 
    Simone Magnani \\ \emailaddr{simone.magnani4@studio.unibo.it} 
    
}

\date{\today}

\begin{document}

\maketitle

\begin{abstract}
    Up to $\sim$2000 characters briefly describing the project.
    
    Non si faccia mancare all'inizio una descrizione anche sommaria di cosa il sistema implementato realizza.
Vista la mole di lavoro dietro al progetto, difficile pensare che i requirement occupino meno di 2-3 facciate: siano più sistematici possibili


Forse non deve essere un abstract?
\end{abstract}

\section{Processo di sviluppo}

% (modalità di divisione in itinere dei task, meeting/interazioni pianificate, modalità di revisione in itinere dei task, scelta degli strumenti di test/build/continuous integration)



\section{Requisiti}

Questa sezione è redatta considerando come utenti i programmatori che utilizzeranno la libreria per sviluppare i giochi, mentre i requisiti dei giocatori che utilizzeranno il software costruito adoperando questa libreria non sono considerati rilevanti.

\subsection{Business}

La libreria consentirà di realizzare giochi da tavolo con le seguenti caratteristiche:
%
\begin{enumerate}
    \item Requisiti di business.
    \begin{enumerate}[label*=\arabic*.]
        \item Giochi basati sul movimento di pedine all'interno di un tabellone;
        \item Giochi a informazione perfetta\footnote{\url{https://it.wikipedia.org/wiki/Gioco_a_informazione_completa}};
        \item Giochi in cui è necessario gestire lo sviluppo della partita (turni, condizioni di vittoria, vincitore della partita, \dots).
    \end{enumerate}
\end{enumerate}

In paricolare deve essere possibile realizzare i seguenti giochi:
%
\begin{enumerate}[resume]
    \item Giochi sviluppati.
    \begin{enumerate}[label*=\arabic*.]
        \item ``\textbf{Put-In-Put-Out}'': un gioco dove l'utente può piazzare e rimuovere una pedina sull'unica casella del tabellone, senza ulteriori funzionalità;
        %
        \item ``\textbf{TicTacToe}'': il gioco del tris, in cui a turno due giocatori piazzano in una casella di una matrice 3x3 il proprio simbolo (solitamente una X e una O).
        %
        Vince il giocatore che riesce a disporre tre delle proprie pedine in linea retta orizzontale, verticale o diagonale.
        %
        Se la matrice viene riempita senza che nessun sia riuscito a creare un tris, il gioco finisce in parità;
        %
        \item ``\textbf{Connect Four}'': Forza4 è un gioco basato su tabellone, in particolare una matrice 6x7, all'interno del quale si possono posizionare delle pedine solo nella prima casella libera di ogni colonna.
        %
        In particolare due giocatori, in modo alternato, inseriscono uno delle proprie pedine all'interno di una colonna.
        %
        Il vincitore è colui che riesce a disporre quattro delle proprie pedine adiacenti in linea retta orizzontale, verticale o diagonale.
        %
        Nel caso di riempimento del tabellone senza che nessun giocatore sia riuscito a vincere, il gioco finisce in parità;
        %
        \item ``\textbf{Othello}'': un gioco basato su una scacchiera 8x8 chiamata othelliera.
        %
        Il gioco prevede la presenza di due giocatori ai quali è assegnato un colore (solitamente bianco e nero). 
        %
        Il gioco inizia con due pedine per giocatore poste nelle caselle centrali a formare una 'X'.
        %
        I turni dei giocatori sono alternati; in ogni turno un giocatore può posizionare una sua pedina in modo da imprigionare una o più pedine dell'avversario tra quella posizionata ed altre sue pedine.
        %
        Ogni pedina imprigionata diventa del giocatore di turno. 
        %
        In mancanza di mosse disponibili il giocatore deve passare il turno.
        %
        L'obiettivo del gioco è di avere più pedine dell'avversario sull'othelliera quando non ci sono più mosse disponibili o quando l'othelliera è piena.
    \end{enumerate}
\end{enumerate}

%------------------------------------------------------------------------------------
\subsection{Utente}

Uno sviluppatore che utilizzerà la libreria dovrà fornire una specifica del gioco utilizzando i costrutti messi a disposizione dalla stessa, e successivamente potrà usufruire del gioco generato da tali specifiche tramite degli appositi comandi.
%
In particolare l'utente può definire i seguenti aspetti:
%
\begin{enumerate}[resume]
    \item Requisiti utente.
    \begin{enumerate}[label*=\arabic*.]
        \item rappresentazione delle pedine all'interno del gioco;
        \item rappresentazione delle celle e del modo in cui compongono il tabellone;
        \item rappresentazione delle mosse;
        \item rappresentazione dello stato del gioco;
        \item regole del gioco:
        \begin{enumerate}[label*=\arabic*.]
            \item con API classica o tramite DSL interno;
            \item generazione delle mosse disponibili a partire da un dato stato di gioco;
            \item definizione delle azioni che vengono eseguite sullo stato del gioco a fronte dell'esecuzione di una mossa;
        \end{enumerate}
        \item descrizione del gioco, che prevede:
        \begin{enumerate}[label*=\arabic*.]
            \item definizione dello stato del gioco iniziale;
            \item definizione di eventuali estensioni del gioco:
            \begin{enumerate}[label*=\arabic*.]
                \item giocatori che partecipano a una partita;
                \item flusso dei turni;
                \item condizioni di terminazione;
            \end{enumerate}
        \end{enumerate}
        \item interazione tramite UI, definendo:
        \begin{enumerate}[label*=\arabic*.]
            \item il tipo di UI utilizzata tra le seguenti:
            \begin{enumerate}[label*=\arabic*.]
                \item \label{req:cli} interfaccia testuale (CLI), fornita dalla libreria;
                \item \label{req:custom_view} interfaccia personalizzata, definita dall'utente;
            \end{enumerate}
            \item \label{req:renderers_parameters} i parametri necessari alla UI per effettuare il display dello stato del gioco;
            \item \label{req:events} i parametri per la conversione degli eventi generati dalla UI in mosse da eseguire sul gioco;
        \end{enumerate}
    \end{enumerate}
\end{enumerate}

\subsection{Funzionali}

La libreria implementerà le funzionalità base che, opportunamente composte, costituiranno un gioco con un determinato insieme di regole.

La libreria dovrà:
%
\begin{enumerate}[resume]
    \item Requisiti funzionali.
    \begin{enumerate}[label*=\arabic*.]
        \item supportare diversi tipi di pedine, a patto che abbiamo uno specifico sopratipo comune;
        \item supportare diversi tipi di caselle, a patto che abbiamo uno specifico sopratipo comune;
        \item \label{req:board_state} supportare diversi tipi di tabelloni:
        \begin{enumerate}[label*=\arabic*.]
            \item fornendo astrazioni di base per i tabelloni comuni (e.g. rettangolari, quadrati, \dots);
            \item lasciando la possibilità allo sviluppatore di rappresentare strutture personalizzate;
        \end{enumerate}
        \item fornire le seguenti azioni sul tabellone:
        \begin{enumerate}[label*=\arabic*.]
            \item inserimento di una pedina su una casella;
            \item rimozione di una pedina da una casella;
            \item lettura dello stato di una casella, ovvero quale eventuale pedina vi è posizionata sopra;
            \item azioni composite, derivate da una sequenza arbitraria delle precedenti;
        \end{enumerate}
        \item gestire lo svolgersi di una partita:
        \begin{enumerate}[label*=\arabic*.]
            \item permettendo di verificare in ogni momento se una mossa è valida;
            \item permettendo di ottenere l'insieme di tutte le mosse valide in un dato istante; 
            \item impedendo l'esecuzione di mosse illegali;
            \item aggiornando lo stato corrente all'esecuzione di una mossa valida, seguendo quanto indicato dalle regole;
            \item \label{req:end_game_cond} rilevando la terminazione della partita;
            \item \label{req:undo} permettendo di annullare una mossa eseguita, mantenendo una cronologia della partita;
        \end{enumerate}
        \item \label{req:dsl} fornire un DSL per la scrittura delle regole, che offre una sintassi semplificata e in linguaggio simil-naturale per:
        \begin{enumerate}[label*=\arabic*.]
            \item \label{req:action} eseguire azioni che modificano lo stato del gioco:
            \begin{enumerate}[label*=\arabic*.]
                \item inserimento di pedine;
                \item rimozione di pedine;
                \item spostamento di pedine;
                \item rimpiazzo di pedine;
                \item avanzamento del turno;
                \item altre azioni personalizzate;
            \end{enumerate}
            \item \label{req:generator} definire le regole per generare le mosse disponibili in un dato momento;
            \item concatenare le istruzioni di base, per creare azioni composite o sequenze di generatori;
            \item iterare sulle proprietà dello stato del gioco;
            \item verificare condizioni sullo stato del gioco;
            \item \label{req:iterating} innestare i costrutti iterativi e condizionali.
        \end{enumerate}
    \end{enumerate}
\end{enumerate}

\subsection{Non funzionali}

Data la natura del software prodotto, la sua dipendenza dal codice degli utenti che ne faranno utilizzo, e la mancanza di elementi distribuiti, non sono individuati requisiti non funzionali.
%
Questi sono lasciati agli utilizzatori della libreria che dovranno calibrarli in base al tipo di applicazione e di utente.

\subsection{Di implementazione}

La libreria dovrà essere sviluppata in Scala, e verranno svolti dei test con \textbf{Scalatest} per minimizzare la presenza di errori e facilitare l'aggiornamento di eventuali funzionalità.


\section{Design architetturale}

% questa sezione deve essere sufficiente per uno sviluppatore per giungere ad un sistema che è organizzato come il nostro

% (architettura complessiva, descrizione di pattern architetturali usati, componenti del sistema distribuito, scelte tecnologiche cruciali ai fini architetturali -- corredato da pochi ma efficaci diagrammi)

% Ricordate che una scelta architetturale può ritenersi giustificata o meno solo a fronte dei requirement che avete indicato; viceversa, ogni requirement "critico" dovrebbe influenzare qualcuna della scelte architetturali effettuate e descritte.
% L'architettura deve spiegare quali sono i sotto-componenti del sistema (da 5 a 15, diciamo), ognuno cosa fa, chi parla con chi e per dirsi cosa -- i diagrammi aiutano, ma poi la prosa deve chiaramente indicare questi aspetti.

Il gruppo ha scelto di lasciare la libreria non vincolata da dipendenze esterne oltre alla scelta del linguaggio Scala, dettata dai requisiti.
%
Non sono infatti state utilizzate librerie esterne se non quelle per il testing automatizzato.

Nel descrivere il design architetturale nelle sotto-sezioni seguenti, viene prima effettuata un'analisi del dominio, successivamente si passa alla descrizione dei pattern architetturali utilizzati.

\subsection{Analisi del dominio}
% requirements, organizzazione del model
L'analisi dei giochi da tavolo ha evidenziato la presenza costante di una divisione in aspetti intensionali e estensionali all'interno della descrizione di un gioco.
%
\paragraph{Aspetti intensionali}
% Game Description, RuleSet, BoardStructure (?)
\paragraph{Aspetti estensionali}
% Game, State, Board
%

\paragraph{Aspetti comuni dei giochi da tavolo}
% Turni
% Giocatori
% Condizioni di terminazione

\subsection{Pattern architetturali}
% descrizione di pattern architetturali usati, MVC
Data la natura del progetto è risultato necessario fornire delle astrazioni ben consolidate.
%
In particolare, per applicazioni che si basano sull'interazione utente, il pattern  \textbf{MVC} è uno dei più noti e flessibili.

% MVC
%Perché, analisi del problema
La separazione delle responsabilità fra i vari componenti è uno dei motivi principali per cui è stato scelto questo pattern: è infatti possibile per un utente della libreria descrivere totalmente un gioco senza fare alcun riferimento alla sua rappresentazione, e solo successivamente collegare ad esso interazione e visualizzazione.
%
Questo è di particolare importanza in una libreria come SBAGS in quanto diversi tipi di interfaccia potrebbero essere necessari per diversi tipi di giochi, e correlare presentazione a modello lo renderebbe più complesso e dispendioso.

Un'ulteriore motivazione risiede nell'abilitare la possibilità di fornire più modalità di visualizzazione del gioco e di interpretazione dell'input utente che possano essere direttamente utilizzabili dagli utenti della libreria. 

\subsection{Architettura complessiva}
%In quale modo
% 0. architettura complessiva, -- model, controller & subcontoller, view & subView:
L'interpretazione del pattern MVC adottata in questo progetto è basata su gerarchie di controller e di view: per ogni fase dell'applicazione - e.g menu, partita in corso - sono presenti diverse coppie di view e controller che gestiscono la specifica situazione.
%
Questi subview e subcontroller sono creati e gestiti dalla \texttt{View} e dal \texttt{Controller} principali, che orchestrano i cambi di scena.

\subsubsection{}
% 1. class diagram

\subsubsection{Interazione View Controller}
% 2. interazione C - V con registrazione/deregistrazione dei listener
%Ogni \texttt{SubView} specifica il proprio tipo di listener (\texttt{SubContoller}). I listener, dopo essere stati aggiunti, vengono notificati in base agli eventi che accadono alla relativa \texttt{SubView}.

% 3. interazione in game -> scambio di messaggi tra GameController e GameView (+ model) -> sequence diagram
%Il \texttt{SubContoller} è in grado di gestire l'interazione con il \texttt{model} 

\section{Design di dettaglio}

%(scelte rilevanti, pattern di progettazione, organizzazione del codice -- corredato da pochi ma efficaci diagrammi)

%Il design di dettaglio "esplode" (dettaglia) l'architettura, ma viene concettualmente prima dell'implementazione, quindi non metteteci diagrammi ultra-dettagliati estratti dal codice, quelli vanno nella parte di implementazione eventualmente.

% Estensioni dello stato tramite type class
% Concetti del DSL (?)
% - usato per MoveGeneration/MoveExecution
% - MoveGeneration usano generators
% - MoveExecution usano actions
% - modifiers per eseguire costrutti iterativi/condizionali e che si basano sui chainables (type class) per astrarre il modo in cui vengono eseguite le iterazioni/condizioni
% - feature (per accedere allo stato nonostante il ruleset sia definito a livello intensionale e quindi non ha accesso a uno stato particolare)
\subsection{RuleSet e Dsl}

Si va ora ad analizzare il desing di dettaglio relativo al \textit{RuleSet} e al suo \textit{Dsl}.

Il \textit{RuleSet} idendifica l'insieme delle regole che definiscono se, in un determinato stato, una \textit{move} è valida e in quale modo questa modifichi lo stato del gioco al momento della sua esecuzione.
% TODO da rivedere
Per separare la logica del \textit{RuleSet} dalle componenti che devono occuparsi della sua creazione si è fatto ricorso ad un builder \textit{RuleSetBuilder}, \textit{un Self-type and mixins}, che funge da coordinatore per raggruppare i vari elementi della generazione del \textit{ruleSet}.

% parlare in questo contesto di tutto quello che riguarda la generazione delle mosse TODO
\subsubsection{MovesGeneration}

\subsubsection{MovesExecution}

\textit{MovesExecution} è la componente che gestisce l'esecuzione delle mosse e ha il compito di definire in quale modo lo stato viene modificato a fronte dell'invocazione di una \textit{move} valida.

\subsection{Interaction}

Le funzionalità abilitanti l'interazione con i giocatori sono state incapsulate totalmente nel modulo \texttt{interaction}, che comprende \textbf{view} e \textbf{controller}, con un interfacciamento verso il \textbf{model}.

Un gioco completo sviluppato con la libreria prevede due viste principali:
\begin{itemize}
    \item \textit{Menu}: il punto di ingresso dell'applicazione dove poter selezionare opzioni come giocare una nuova partita oppure uscire dal programma;
    \item \textit{Game}: una partita in corso.
\end{itemize}
%
Ciascuna di queste è composta da una specifica coppia di \texttt{subView} e \texttt{subController}.
%

% - View
\subsubsection{View}
% - - View principale -> Gestione subview
La \texttt{View} principale è in grado di creare e inizializzare le \texttt{SubView} esponendo un metodo per ritornare ognuna di esse: questo permette di centralizzare le informazioni relative al tipo della \texttt{View} e consente di rimpiazzarla in blocco. %brutto
% - - Menu view
\paragraph{Menu view} 
%
È il componente per gestire tutto ciò che è precedente o successivo al \textit{Game} (e.g. creazione \textit{Game}, chiusura dell'applicazione).
%
Il \texttt{MenuView} permette tramite un'apposita UI di scegliere come procedere tra le varie opzioni.
% - - Game view
\paragraph{Game view} 
%
È in grado di presentare il \textit{Game} nel caso di una \textit{move} valida, e in caso contrario gestisce la \texttt{Failure}.
%
Nello specifico, tramite la composizioni di \texttt{renderer}, la \texttt{GameView} presenta la grafica dello stato attuale del \textit{Game}.
% - - - Renderers
\subparagraph{Renderer}
%
Compongono le \texttt{GameView} ed espongono un metodo per disegnare una specifica caratteristica del \textit{Game}.
%
Richiamando l'aggiornamento di tutti i \texttt{renderer} la \texttt{GameView} è in grado di mostare tutte le caratteristiche dello stato attuale del \textit{Game}.
% - - - Event
\subparagraph{Event}
%
Rappresentano un'interazione dell'utente con la \texttt{GameView} e un'insieme di \texttt{Event} può comporre una \textit{move}.%va scritto qui o sotto?
%
Una \texttt{GameView}, in quanto \texttt{Observable}, è in grado di notificare il proprio \texttt{GameController} degli \texttt{Event} che vengono generati. 
% - Controller
\subsubsection{Controller}
I controller presentano una dipendenza dalle view in quanto devono chiamare i metodi della specifica \texttt{subView} per fornire il feedback appropriato rispetto all'interazione intrapresa dal giocatore.
%
% - - Application controller
L'\texttt{ApplicationController} è responsabile di gestire la \texttt{View} principale, aggiungendovi i \textbf{listener} relativi e facendola partire.
% - - Menu controller
\paragraph{Menu controller}
%
Presenta le funzionalità per gestire l'avvio di una partita e la terminazione dell'applicazione.
%
Questo controller è una prima interfaccia con il \textbf{model}, che nel caso di avvio di una partita viene utilizzato nella forma della \texttt{GameDescription} per la generazione di un nuovo \texttt{Game}.
%
Successivamente viene preparata la GameView con i parametri relativi, viene aggiunto un nuovo \texttt{GameController} come listener e si passa alla nuova vista.
% - - Game controller
\paragraph{Game controller}
%
Gestisce gli \texttt{Event} che gli vengono notificati, presentando un comportamento diverso in base alla situazione:
\begin{itemize}
    \item Se l'evento assieme a quelli precedentemente salvati corrisponde ad una \texttt{Move}, allora questa viene eseguita sul \texttt{Game} e gli eventi memorizzati vengono azzerati;
    \item Se l'evento assieme a quelli precedentemente salvati non corrisponde ad una \texttt{Move}, allora viene memorizzato;
    \item Se l'evento è \texttt{Quit} allora termina la \texttt{GameView}.
\end{itemize}


\section{Implementazione}

La seguente sezione va ad analizzare quelli che sono gli aspetti e le decisioni implementative che caratterizzano il codice.
%
Si riportano solo i casi degni di nota e si lascia il resto al codice e alla documentazione su di esso che sono stati pensati per essere autoesplicativi e semplici da comprendere per un utilizzatore della libreria.

% ---------------------------------------------------

\subsection{Passaggio implicito delle type class}

Come anticipato nella Sezione \ref{sec:detailed_design}, il meccanismo delle type class è stato sfruttato molto frequentemente.
%
Al fine di massimizzare la pulizia e la leggibilità del codice, si è deciso di utilizzare il passaggio implicito dei parametri per evitare di dover comunicare esplicitamente le istanze delle type class.

% ---------------------------------------------------

\subsection{Sintassi del DSL}

Il DSL è stato scritto quanto più auto-esplicativo possibile, in modo da mettere l'utente della libreria nella condizione di dover fare il minor sforzo possibile per la scrittura delle regole di un gioco.
%
Questo ci ha portato alla scrittura di un DSL molto simile (ove possibile) al linguaggio naturale e che richiede una conoscenza estremamente limitata del linguaggio Scala.

Per ottenere questo risultato è stato necessario sfruttare alcune funzionalità offerte da Scala:
\begin{itemize}
  \item \textbf{metodi come operatori infissi}, che richiedono quindi che un metodo sia preceduto e seguito da un oggetto;
  \item \textbf{conversioni implicite}, per aggiungere metodi a tipi già esistenti o per adattare oggetti in maniera trasparente;
  \item \textbf{uso delle parentesi graffe al posto delle parentesi tonde}, per avere una strutturazione più familiare nel caso di nesting dei costrutti.
\end{itemize}

I meccanismi citati, specialmente il primo, richiedono spesso l'adattamento della struttra sintattica delle frasi, scritte in linguaggio naturale, per fare in modo che sia supportata a livello sintattico anche da Scala.
%
In particolare, si è cercato di adattare il più possibile le frasi, del DSL, per fare in modo che la struttura potesse essere ricondotta a sequenze del tipo \texttt{object method object method object ...}, in modo da evitare la presenza di parentesi.

Nel Listato \ref{lst:dsl_example} è mostrato uno snippet di codice che mostra a grandi linee le modalità con cui i meccanismi di scala sono stati adottati nella scrittura del costrutto iterativo.
%
\lstinputlisting[language=scala, caption={Dichiarazione della sintassi di iterazione e esempio di utilizzo}, label=lst:dsl_example]{code/dsl.scala}
%
Un'istruzione di iterazione parte dalla parola chiave \texttt{iterating}, ovvero un oggetto il cui unico metodo è \texttt{over}.
%
Questo metodo accetta una \texttt{Feature} che estrae dallo stato un \texttt{Traversable[F]} su cui deve essere eseguita l'iterazione.
%
Come anticipato nella Sezione \ref{sec:dsl_design}, è necessario usare una \texttt{Feature} poichè le regole vengono definite a livello intensionale, quindi non si ha accesso diretto allo stato.

Il metodo \texttt{over} restituisce un nuovo oggetto di tipo \texttt{Iteration[F]}, che a sua volta contiene un unico metodo \texttt{as} al quale deve essere passata la funzione di iterazione.
%
Questa è una funzione che associa ad ogni elemento estratto dallo stato un nuovo oggetto di tipo \texttt{T} generico.
%
Per fare in modo che l'iterazione sia possibile, è necessario anche che sia implicitamente definita un'istanza di \texttt{Chainable}, per stabilire in che modo concatenare le \texttt{T} restituite dalle singole iterazioni e quale deve essere il valore di partenza per l'accumulazione.

Infine viene mostrato un esempio di utilizzo del costrutto di iterazione su una proprietà \texttt{Traversable} dello stato.
%
L'esempio utilizza il metodo \texttt{generate} fornito dalla libreria, che restituisce un oggetto di tipo \texttt{Generator}, per cui la libreria fornisce già un istanza di \texttt{Chainable}.

% ---------------------------------------------------

\subsection{Cli}
Data l'infrastruttura di gestione dell'interazione con il giocatore, un utente della libreria può scegliere se scrivere una propria implementazione dell'interfaccia grafica oppure se avvalersi dell'interfaccia a riga di comando fornita dalla libreria.
%
Questa opzione è compatibile solamente con i giochi aventi una \textit{Board} \textbf{rettangolare}, per tutti gli altri è necessario fornire un'implementazione alternativa.

L'interfaccia a riga di comando è generata a partire dalle caratteristiche del gioco e dai \texttt{Renderer} associati, seguendo così un approccio standardizzato per associare a diverse \textbf{estensioni} del gioco una rappresentazione in maniera automatica.
%
Ciò consente di avere una visualizzazione semplice del gioco scritto usando la libreria con il minimo sforzo da parte dell'utente, che ha però la possibilità di personalizzare la grafica fornendo una serie di parametri oppure nuove implementazioni dei singoli componenti per i quali desidera un aspetto diverso.
%
In particolare gli aspetti personalizzabili per il \texttt{BoardRenderer} sono:
\begin{itemize}
  \item Rappresentazione degli indici delle colonne;
  \item Rappresentazione degli indici delle righe;
  \item Separatore degli elementi grafici;
  \item Terminazione riga;
  \item Rappresentazione dei \textit{Tile} vuoti;
  \item Rappresentazione dei diversi \textit{Pawn}.
\end{itemize}
%
Ognuno di questi parametri ha un valore di default, seguendo il principio della \textbf{convenzione} prima della \textbf{personalizzazione} in modo da snellire il processo di preparazione dell'interfaccia da parte dell'utente qualora le impostazioni base fossero sufficienti.
%
Infine, i \texttt{Renderer} rimanenti non sono parametrici in quanto sufficientemente semplici da poter essere completamente sostituiti in maniera rapida da nuove implementazioni.

% ---------------------------------------------------

\subsection{Configurazione dell'app: GameSetup}

La configurazione degli elementi che gestiscono l'interazione tra un giocatore umano e l'applicazione finale richiede svariati passaggi.
%
\'E necessario che lo sviluppatore configuri correttamente i parametri di rendering, la comunicazione tra view e controller e altri dettagli spesso dipendenti l'uno con l'altro.

Per porre rimedio a questo problema e per fornire un singolo punto di configurazione, è stato creato un piccolo modulo di utility -- il \texttt{GameSetup} -- che ha l'obiettivo di aiutare lo sviluppatore fornendo uno scheletro della configurazione, implementato con valori di default sensati.
%
Questo modulo consente a chi utilizza la libreria di fornire la propria configurazione personalizzata facendo override dei metodi presenti nel \texttt{GameSetup} in stile \textbf{template method}.

La libreria supporta al momento una configurazione dettagliata solo per l'interfaccia testuale (estendendo il trait \texttt{CliGameSetup}), ed in particolar modo aiuta lo sviluppatore nel caso venga utilizzata una board di tipo rettangolare (aggiungendo il mixin \texttt{RectangularBoardSetup}).
%
Le principali funzionalità che sono state fornite sono:
%
\begin{itemize}
  \item Scelta dei renderer che dovranno essere utilizzati a partire da un \textbf{builder}, per poter scegliere dinamicamente cosa disegnare;
  \item Configurazione del parsing dei comandi a partire da un \textbf{builder}, per aggiungere dinamicamente comandi con sintassi arbitrarie;
  \item Sintassi semplificata per i metodi di entrambi i builder, che sono internamente implementati come template method che richiamano i metodi potenzialmente sovrascrivibili dallo sviluppatore stesso.
\end{itemize}

Una volta scritta la propria configurazione nel \texttt{GameSetup}, è possibile avviare l'applicazione chiamando \texttt{AppRunner.run} fornendo la propria istanza di \texttt{GameSetup}, che viene generalmente chiamato dall'interno dell'entry point dell'app.

% ---------------------------------------------------

\subsection{Giochi}
Oltre alla libreria stessa sono stati sviluppati alcuni giochi per fungere da esempi per gli utenti e dimostrare l'utilizzo delle funzionalità proposte.
%
Per realizzare un gioco utilizzando la libreria è necessario definire i \textit{Pawn}, le \textit{Move}, la \textit{Board} e lo \textit{State}.% lo state è dinamico
%
Solitamente questo viene fatto tramite \texttt{Trait}, \texttt{Case Class} e \texttt{Case Object}.
%
La fase successiva prevede, in un qualsiasi ordine, di definire:
\begin{itemize}
  \item la \texttt{GameDescription}, come \texttt{object} che estende \texttt{GameDescription}, dove sono dichiarati gli \texttt{implicit} necessari al funzionamento delle \texttt{extension};
  \item il \textit{RuleSet}, che consiste in un \texttt{object} che estende il \texttt{RuleSet} e può essere descritto tramite il \textbf{DSL} se \textbf{mixato} con il \texttt{RuleSetBuilder}
\end{itemize}

Infine è necessario sviluppare un \textbf{main} che sia in grado di eseguire l'applicazione così definita.
%
Per fare ciò ci si può avvalere del \texttt{GameSetup}, o più in particolare del \texttt{CliGameSetup} se si vuole usufruire della cli già fornita, dove vengono definiti i \texttt{Renderer} ed i parametri del \textbf{controller}.
%
Dato un \texttt{GameSetup} è possibile eseguire il gioco semplicemente utilizzando l'\texttt{AppRunner}.

% ---------------------------------------------------

\subsection{Testing}
I test effettuati sul codice sviluppato hanno lo scopo principalmente di garantire la qualità del codice, di favorire il cambiamento ed infine di documentazione del software sviluppato, seguendo la \textbf{quality school} e la \textbf{agile school} come filosofie di riferimento.
%
Avendo approcciato il progetto con la metodologia \textbf{TDD} la maggior parte del codice risulta avere degli unit test che coprono le singole funzionalità.
%
Ci sono alcune eccezioni, ad esempio l'interfaccia testuale risulta essere poco coperta dai test a causa della necessità di acquisire input da tastiera e la scarsa utilità di testare i risultati di stampe a video.
%
La copertura risulta invece molto alta nel \texttt{model}, arrivando ad avere il 100\% di coverage per il \texttt{core}.
%
Oltre agli \textbf{unit test} utilizzati per l'approccio TDD e sviluppati prima del codice stesso, sono presenti anche degli \textbf{integration test}, aggiunti una volta terminato lo sviluppo del codice di una singola unità per assicurare la corretta interazione con le altre.
%
Infine, i \textbf{system test} sono stati effettuati nei giochi d'esempio, che forniscono un ambiente articolato e completo dove poter testare l'interazione fra i diversi moduli del sistema.

\subsubsection{Test doubles}
Ove possibile sono stati effettuati test di tipo funzionale e \textbf{blackboxed}, mentre dove è risultato necessario sono stati effettuati dei test strutturali \textbf{whiteboxed}.
%
Entrambi le casistiche hanno visto un impiego frequente dei \textbf{test doubles} sfruttando la libreria \textbf{scalamock} per rimuovere le dipendenze dagli unit test o per verificare il comportamento interno di un componente nel caso di test whitebox.

\subsubsection{Stile dei test}
Lo stile adottato è stato \textbf{FlatSpec} nella quasi totalità dei casi, in quanto il più adatto agli unit test e semplice sia da consultare che da modificare, a favore di uno sviluppo agile del software a fronte di cambiamenti nei requisiti o nella loro comprensione da parte del team.

% ---------------------------------------------------

\subsection{Divisione del lavoro}
%parti comuni:
% - core
% - extension
% - divisione in due sottogruppi per la gestione delle due macro-componenti del progetto
% -- Comuni: RuleSet, DSL, Features
La divisione del lavoro del team si può suddividere in tre stadi:
%
Inizialmente il gruppo ha affrontato l'analisi del dominio e l'analisi architetturale dell'intero sistema in completa collaborazione, tramite meeting intensi nei quali si è cercato di delineare quelli fossero gli obiettivi del team nel breve e lungo periodo.
%
Una volta stabilito quello che è il design architetturarale del sistema si è scelto di dividere il team in due sotto-team di due persone ciascuno.
%
Il primo gruppo è quello formato da Dente ed Evangelisti al quale è stato assegnato il compito di lavorare sul \textit{rule set} e sul DSL relativo.
%
Il secondo gruppo formato da Magnani e Nemati si è occupato principalemente della parte di interaction.
%
In fine, all'interno di ogni gruppo, dopo aver effettuato un'attenta analisi in dettaglio delle sezioni di interesse, si è suddiviso il lavoro per singoli membri.

Le divisioni del lavoro svolte sono state sempre ben documentate nel corso dello svolgimento del progetto ma è da tenere in considerazione che la stretta collaborazione e il continuo ciclo di daily scrum ha fatto in modo che tutti i membri del gruppo fossero sempre tenuti al corrente dell'andamento del progetto.
%
Si sottolinea che buona parte del codice è stata scritta in \textbf{pair programming} per massimizzare la qualità del codice e ridurre il tempo relativo alla revisione del codice.

\subsubsection{Dente Francesco}

La porzione principale del progetto di cui mi sono occupato è stata la scrittura e la progettazione del DSL per la definizione del \textit{rule set}, svolta in collaborazione con Evangelisti.
%
In particolare, nonostante gran parte del lavoro sia stata portata avanti cooperativamente, ho dato un contributo particolare nella scrittura del codice relativo alla generazione delle mosse e all'utilizzo del concetto di Chainable all'interno dei costrutti iterativi e condizionali.

Per quanto riguarda invece la parte core della libreria, mi sono occupato dell'introduzione delle type class per implementare le estensioni dello stato, nonostante queste siano state poi aggiornate con le successive iterazioni in maniera cooperativa da tutti i membri del gruppo.

Un'altra area su cui mi sono concentrato è stata l'implementazione delle astrazioni di GameSetup e in generale della gestione della fase di configurazione e avvio dell'applicazione.

Infine, mi sono occupato dello sviluppo e del testing del gioco d'esempio Othello, che è stato implementato una volta terminata la scrittura del codice della libreria in sè.

\subsubsection{Evangelisti Davide}
% dsl in comune con Dente, RuleSet (con particolare attenzione alla move execution e le Action), PutInPutOut Revisited, 'after each move', RuleSetBuilder
% RuleSetBuilder
% MovesExecution
Mi sono occupato principalemente della scrittura del \textit{rule set} assieme al collega Dente con il quale ho affrontato l'analisi del problema e la progettazione di massima del DSL.
%
La parte, del \textit{rule set} alla quale mi sono dedicato maggiormente è stata quella relativa all'esecuzione delle \textit{move} e al relativo DSL, che permette all'utente della libreria di scrivere facilmente in quale modo lo stato del gioco viene modificato da un'azione.
%
Nello specifico mi sono occupato del trait \texttt{MovesExecution}, nel quale, tramite le funzionalità offerte da Scala, ho costruito la parte di DSL relativa all'esecuzione delle \textit{move}.
%
Grazie alle conversioni implicite sono riuscito a rendere la \texttt{MovesExecution} indipendente dal sistema delle \texttt{Actions} e allo stesso tempo queste ultime si integrano perfettamente all'interno del sistema di creazione delle \textit{move}.

Per quanto riguarda il \texttt{RuleSet} e il \texttt{RuleSetBuiledr} ho scelto di dividere questi due aspetti in modo da rendere il \texttt{RuleSet} indipendente dal modo in cui questo è generato.
%
Così facendo do la possibilità a chi utilizza la libreria di scegliere se utilizzare il DSL oppure approciarsi alla generazione e all'esecuzione delle mosse in un modo diverso.

Oltre alla parte relativa al \textit{rule set} mi sono occupato anche della \textit{Board} e della sua scrittura.
%
In questo particolare frangente non sono presenti tecniche di programmazione avanzate se non l'immutabilità della \textit{Board} stessa.
%
Si è scelto di mantenere la \textit{Board} immutabile in vista di modifiche future e di funzionalità aggiuntive della libreria, questa scelta si è dimostrata corretta in quanto durante la scrittura della funzionalità opzionale di \textbf{undo}, ovvero la possibilità di tornare allo stato precedente, la presenza di un oggetto immutabile semplifica enormemente il lavoro.

Oltre a questo mi sono occupato dello sviluppo e del mantenimento del primo gioco sviluppato dal gruppo, ovvero, PutInPutOut.


\subsubsection{Magnani Simone}
% interaction in comune con Nemati. In singolo:
Dopo lo sviluppo delle parti comuni e la divisione in sottogruppi, insieme a Nemati, abbiamo lavorato allo sviluppo di tutto ciò che riguarda l'interazione con l'utente.
% View - GameView
In particolare, occupandomi della \texttt{View} ho cercato di rendere la generazione di ogni componente il più lontano possibile dal \texttt{model}, tanto che per giochi con \textit{Board} rettangolari esiste un'implementazione automatica della visualizzazione ed interazione.
% renderer
Ho sviluppato i \texttt{Renderer} in modo da comporre la \texttt{GameView}, ma incapsulando l'obiettivo specifico di ogni \texttt{Renderer}.
%
Questo ha reso la \texttt{GameView} indipendente dalle varie \texttt{extension} di ogni particolare gioco permettendo la generazione automatica della stessa.
% Connect4
Infine mi sono occupato del testing e dell'implementazione di \textbf{Connect Four}, il che mi ha portato a realizzare delle \texttt{apply} di utility per le \texttt{extension}.

\subsubsection{Nemati Shapour}
% interaction in comune con Magnani. In singolo:
% Controller
% Event
% Input parser
% Rectangular Board
% Tic-Tac-Toe


\section{Restrospettiva}

%(descrizione finale dettagliata dell'andamento dello sviluppo, del backlog, delle iterazioni; commenti finali)

%Si noti che la retrospettiva è l'unica sezione che può citare aneddoti di cosa è successo in itinere, mentre le altre sezioni fotografino il risultato finale. Se gli studenti decideranno (come auspicato) di utilizzare un product backlog e/o dei backlog delle varie iterazioni/sprint, è opportuno che questi siano file testuali tenuti in versione in una cartella "process", così che sia ri-verificabile a posteriori la storia del progetto.

\subsection{Sprint 1}
Il primo sprint è stato principalmente incentrato sul setup del progetto, l'analisi del problema e la creazione delle interfacce core della libreria, questo ha portato a meeting molto frequenti e prolungati durante la giornata.
\paragraph{Svolgimento e sviluppo dello sprint}
Sono stati inizializzati i tool inerenti al processo di sviluppo, in particolare:
\begin{itemize}
   \item \textbf{SBT} per la gestione delle dipendenze e delle build;
   \item \textbf{Travis CI} per il testing automatico e la Continous Integration.
\end{itemize}
Si è ritenuto necessario sin da subito creare convenzioni comuni riguardo i termini, la documentazione e la scrittura di codice, ad esempio l'uso di un glossario comune.
%
Dopo una prima parte di Design, sono stati scritti i test e le relative interfacce necessarie alla creazione del gioco PutInPutOut; infine sono stati creati i test del gioco tramite i requisiti specificati ed è stato implementato il gioco.
\paragraph{Considerazioni finali}
Lo sprint non ha sollevato particolari problemi ed è terminato nei tempi previsti.

Una possibile prosecuzione prevede:
\begin{itemize}
  \item la possibilità di avere dei \textbf{turni} e dei giocatori all'interno della partita;
  \item un'interfaccia più facilmente utilizzabile per l'utente della libreria;
  \item l'implementazione del gioco \textbf{Tic-Tac-Toe}.
\end{itemize}

\subsection{Sprint 2}
Il secondo sprint è stato principalmente incentrato sull'estensione delle funzionalità core della libreria, lo sviluppo di utility comunemente adottate dai giochi target della libreria e la creazione del gioco Tic-Tac-Toe.
\paragraph{Svolgimento e sviluppo dello sprint}
Lo sviluppo della libreria è stato parallelizzato alla creazione di Tic-Tac-Toe.
%
Alcuni dei requisiti di Tic-Tac-Toe sono stati resi più generali al fine di integrarli nella libreria: in particolare la necessità di avere dei \textit{player}, un \textit{game flow} basato su turni e su \textit{game ending conditions}.
\paragraph{Considerazioni finali}
Durante lo sprint si sono riscontrati problemi con l'utilizzo di \textbf{ScalaMock} e, data la complessità del progetto, il team si è trovato a lavorare sulle stesse funzionalità, portando a una minore parallelizzazione dei task e, di conseguenza, a un minor rigore nel processo di sviluppo.

Una possibile prosecuzione prevede:
\begin{itemize}
  \item un'interfaccia più facilmente utilizzabile per l'utente della libreria;
  \item l'implementazione di una interfaccia visuale del gioco \textbf{Tic-Tac-Toe}.
\end{itemize}

\subsection{Sprint 3}
Il terzo sprint aveva inizialmente lo scopo di sviluppare un primo DSL ed un'interfaccia utente testuale, ma a causa di rigidità del codice precedente è stato necessario operare un \textbf{refactor} su una grande porzione di codice, e come risultato non è stato possibile raggiungere gli obiettivi prefissati.
\paragraph{Considerazioni finali}
Dal punto di vista del processo di sviluppo la metodologia \textbf{Scrum} non è stata seguita in toto, in particolare i Daily Scrum non sempre hanno avuto luogo seguendo gli step prestabiliti, ma si è proceduto in gruppo a lavorare sul codice da rifattorizzare.

Una possibile prosecuzione prevede il completamento dei Task ancora aperti a cui va ad aggiungersi lo sviluppo del gioco \textbf{Connect Four}.

\subsection{Sprint 4}
Il quarto Sprint ha raggiunto con successo gli obiettivi posti: lo sviluppo di un primo DSL, di un'interfaccia utente testuale, e del gioco \textbf{Connect Four}.
\paragraph{Considerazioni finali}
La metodologia \textbf{Scrum} è stata applicata efficacemente, correggendo gli errori fatti negli Sprint precedenti.
%
Una possibile miglioria riguardante gli strumenti a supporto del processo di sviluppo è quella di individuare dei \textbf{Task} a grana più fine ed inserirli in \textbf{Trello}, al fine di guadagnare in autonomia dei membri del team, e di ottenere una maggiore chiarezza su quali funzionalità siano già pronte, quali in sviluppo, e quali totalmente assenti.

Una possibile prosecuzione prevede di migliorarela qualità del codice, ed aggiungere alcune funzionalità in più per quanto riguarda il DSL e l'interfaccia testuale.

\subsection{Sprint 5}
Il quinto Sprint ha portato gli incrementi attesi, fornendo un DSL più potente ed una view più dinamica.
\paragraph{Considerazioni finali}
A causa di impegni personali di alcuni membri del team non sempre si sono tenuti i \textbf{Daily Scrum} con la presenza di tutto il team, ma oltre questo piccolo inconveniente non si riscontrano altre problematiche di processo.

Una possibile prosecuzione prevede lo sviluppo del gioco \textbf{Othello}, la stesura del \textbf{Report}, e l'aggiunta di funzionalità dichiarate opzionali durante la definizione dei requisiti.

\subsection{Sprint 6}
%
Il sesto Sprint ha portato gli incrementi attesi sia dal punto di vista del codice che della documentazione: è stato, infatti, sviluppato il gioco \textbf{Othello} e il \textbf{Report} è quasi completato, inoltre è stata prodotta la \textbf{Scaladoc} mancante per il codice.
%
\paragraph{Considerazioni finali}
%
Si è ritenuto necessario allungare la scadenza di questo sprint in quanto, per impegni personali, non è stato possibile dedicare tutto il tempo necessario per completare gli obiettivi dello sprint.u

\section*{Stylistic Notes}

Use a uniform style, especially when writing formal stuff: $X$, X, $\mathbf{X}$, $\mathcal{X}$, \texttt{X} are all different symbols possibly referring to different entities. 

This is a very short paragraph.

This is a longer paragraph (notice the blank line in the code).
It composed by several sentences.
%
You're invited to use comments within \texttt{.tex} source files to separate sentences composing the same paragraph.

Paragraph should be logically atomic: a subordinate sentence from one paragraph should always refer to another sentence from within the same paragraph.

The first line of a paragraph is usually indented.
%
This is intended: it is the way \LaTeX{} lets the reader know a new paragraph is beginning.

Use the \href{https://en.wikibooks.org/wiki/LaTeX/Source_Code_Listings}{\texttt{listing}} package for inserting scripts into the \LaTeX{} source.

\nocite{*} % Includes all references from the `references.bib` file
\bibliographystyle{plain}
\bibliography{references}

\end{document}
