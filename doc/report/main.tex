\documentclass{scrartcl}
\usepackage[utf8]{inputenc}
\usepackage{hyperref}
\usepackage{url}
\usepackage{natbib}
\usepackage{graphicx}

\newcommand{\emailaddr}[1]{\href{mailto:#1}{\texttt{#1}}}

\title{\LARGE
    PPS Final Report
}

\author{
    Davide Evangelisti \\ \emailaddr{davide.evangelisti2@studio.unibo.it}
    \and 
    Francesco Dente \\ \emailaddr{francesco.dente@studio.unibo.it} 
    \and 
    Shapour Nemati \\ \emailaddr{shapour.nemati@studio.unibo.it}
    \and 
    Simone Magnani \\ \emailaddr{simone.magnani4@studio.unibo.it} 
    
}

\date{\today}

\begin{document}

\maketitle

\begin{abstract}
    Up to $\sim$2000 characters briefly describing the project.
    
    Non si faccia mancare all'inizio una descrizione anche sommaria di cosa il sistema implementato realizza.
Vista la mole di lavoro dietro al progetto, difficile pensare che i requirement occupino meno di 2-3 facciate: siano più sistematici possibili


Forse non deve essere un abstract?
\end{abstract}

\section{Processo di sviluppo}

(modalità di divisione in itinere dei task, meeting/interazioni pianificate, modalità di revisione in itinere dei task, scelta degli strumenti di test/build/continuous integration)


\section{Requisiti}

Attenzione in particolare ai requirement non funzionali: 1) non siano troppo vaghi altrimenti sono inverificabili, e quindi praticamente inutili; 2) se il sistema è distribuito, è inevitable dire cosa vi aspettate in termini di di robustezza a cambiamenti/guasti (quali?, come?), e scalabilità (in quale dimensione? fino a che punto?).


\subsection{Business}
\subsection{Utente}
\subsection{Funzionali}
\subsection{Non funzionali}
\subsection{Di implementazione}


\section{Design architetturale}

questa sezione deve essere sufficiente per uno sviluppatore per giungere ad un sistema che è organizzato come il nostro

(architettura complessiva, descrizione di pattern architetturali usati, componenti del sistema distribuito, scelte tecnologiche cruciali ai fini architetturali -- corredato da pochi ma efficaci diagrammi)

Le scelte tecnologiche non dovrebbero essere anticipate troppo per ovvi motivi.. prima le fate prima impattano tutta la parte successiva e quindi diventano più difficilmente riconsiderabili (comunque in linea di principio ogni scelta ha una sua posizione logica precisa, e potrebbe essere nei requirement fino all'implementazione).

Ricordate che una scelta architetturale può ritenersi giustificata o meno solo a fronte dei requirement che avete indicato; viceversa, ogni requirement "critico" dovrebbe influenzare qualcuna della scelte architetturali effettuate e descritte.
L'architettura deve spiegare quali sono i sotto-componenti del sistema (da 5 a 15, diciamo), ognuno cosa fa, chi parla con chi e per dirsi cosa -- i diagrammi aiutano, ma poi la prosa deve chiaramente indicare questi aspetti.


\section{Design di dettaglio}

(scelte rilevanti, pattern di progettazione, organizzazione del codice -- corredato da pochi ma efficaci diagrammi)

Il design di dettaglio "esplode" (dettaglia) l'architettura, ma viene concettualmente prima dell'implementazione, quindi non metteteci diagrammi ultra-dettagliati estratti dal codice, quelli vanno nella parte di implementazione eventualmente.


\section{Implementazione}

questa sezione deve essere sufficiente per uno sviluppatore per avere una implementazione, essenzialmente equivalente alla nostra

(per ogni studente, una sotto-sezione descrittiva di cosa fatto/co-fatto e con chi, e descrizione di aspetti implementativi importanti non già presenti nel design)


L'implementazione "esplode" il design, ma solo laddove pensiate che serva dire qualcosa.

Cercate di dare una idea di quanto pensate che i vostri test automatizzati coprano il codice e dove: è importante per stimare il potenziale impatto di una modifica al software.


\section{Restrospettiva}
(descrizione finale dettagliata dell'andamento dello sviluppo, del backlog, delle iterazioni; commenti finali)

Si noti che la retrospettiva è l'unica sezione che può citare aneddoti di cosa è successo in itinere, mentre le altre sezioni fotografino il risultato finale. Se gli studenti decideranno (come auspicato) di utilizzare un product backlog e/o dei backlog delle varie iterazioni/sprint, è opportuno che questi siano file testuali tenuti in versione in una cartella "process", così che sia ri-verificabile a posteriori la storia del progetto.


\section*{Stylistic Notes}

Use a uniform style, especially when writing formal stuff: $X$, X, $\mathbf{X}$, $\mathcal{X}$, \texttt{X} are all different symbols possibly referring to different entities. 

This is a very short paragraph.

This is a longer paragraph (notice the blank line in the code).
It composed by several sentences.
%
You're invited to use comments within \texttt{.tex} source files to separate sentences composing the same paragraph.

Paragraph should be logically atomic: a subordinate sentence from one paragraph should always refer to another sentence from within the same paragraph.

The first line of a paragraph is usually indented.
%
This is intended: it is the way \LaTeX{} lets the reader know a new paragraph is beginning.

Use the \href{https://en.wikibooks.org/wiki/LaTeX/Source_Code_Listings}{\texttt{listing}} package for inserting scripts into the \LaTeX{} source.

\nocite{*} % Includes all references from the `references.bib` file
\bibliographystyle{plain}
\bibliography{references}

\end{document}
